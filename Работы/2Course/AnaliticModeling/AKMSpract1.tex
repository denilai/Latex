\documentclass[a4paper,14pt]{extarticle}

% Путь до папки с общими шаблонами
\newcommand{\pathToCommonFolder}{/home/denilai/Documents/repos/latex/Common}
% Название работы в титуле
\newcommand{\workname}{Отчет по практической работе №1}
% Название дисциплины в титуле
\newcommand{\discipline}{Анализ и концептуальное моделирование систем}
% Название кафедры в титуле
\newcommand{\kafedra}{Кафедра Практической и Прикладной Информатики}
% Тема работы в титуле
\newcommand{\theme}{Описание функционала системы}
% Должность преподавателя в титуле
\newcommand{\rang}{доцент}
% ФИО преподавателя в титуле
\newcommand{\teacherfio}{В.~В.~Пяткин}


\usepackage{tabularx}



\usepackage{booktabs}
\newcolumntype{b}{X}
\newcolumntype{s}{>{\hsize=.5\hsize}X}
\newcommand{\heading}[1]{\multicolumn{1}{c}{#1}}

% установка размера шрифта для всего документа
%\fontsize{20pt}{18pt}\selectfont
\usepackage{extsizes} % Возможность сделать 14-й шрифт

% Вставка заготовки преамбулы
% Этот шаблон документа разработан в 2014 году
% Данилом Фёдоровых (danil@fedorovykh.ru) 
% для использования в курсе 
% <<Документы и презентации в \LaTeX>>, записанном НИУ ВШЭ
% для Coursera.org: http://coursera.org/course/latex .
% Исходная версия шаблона --- 
% https://www.writelatex.com/coursera/latex/5.3

% В этом документе преамбула

% Для корректного использования русских символов в формулах
% пакеты hyperref и настройки, связанные с ним, стоит загуржать
% перед загрузкой пакета mathtext



% поддержка русских букв
% кодировка шрифта
%\usepackage[T2A]{fontenc} 
\usepackage{pscyr}

% использование ненумеровонного абзаца с добавлением его в содержаниеl

\newcommand{\anonsection}[1]{\section*{#1}\addcontentsline{toc}{section}{#1}}
\newcommand{\sectionunderl}[1]{\section*{\underline{#1}}}


% настройка окружения enumerate
\usepackage{enumitem}
\setlist{noitemsep}
\setlist[enumerate]{labelsep=*, leftmargin=1.5pc}

\usepackage{hyperref}

% сначала ставить \usepackage{extsizes} % Возможность сделать 14-й шрифт
% для корректной установки полей вставлять преамбулу следует в последнюю очередь (но перед дерективой замены \rmdefault)
\usepackage[top=20mm,bottom=25mm,left=35mm,right=15mm]{geometry} % Простой способ задавать поля

\hypersetup{				% Гиперссылки
	unicode=true,           % русские буквы в раздела PDF
	pdftitle={Заголовок},   % Заголовок
	pdfauthor={Автор},      % Автор
	pdfsubject={Тема},      % Тема
	pdfcreator={Создатель}, % Создатель
	pdfproducer={Производитель}, % Производитель
	pdfkeywords={keyword1} {key2} {key3}, % Ключевые слова
	colorlinks=true,       	% false: ссылки в рамках; true: цветные ссылки
	linkcolor=red,          % внутренние ссылки
	citecolor=black,        % на библиографию
	filecolor=magenta,      % на файлы
	urlcolor=blue           % на URL
}

%%% Работа с русским языком
\usepackage{cmap}					% поиск в PDF
\usepackage{mathtext} 				% русские буквы в формулах
\usepackage[T2A]{fontenc}			% кодировка
\usepackage[utf8]{inputenc}			% кодировка исходного текста
\usepackage[english,russian]{babel}	% локализация и переносы
\usepackage{indentfirst}
\frenchspacing

%для изменения названия списка иллюстраций
\usepackage{tocloft}


\renewcommand{\epsilon}{\ensuremath{\varepsilon}}
\renewcommand{\phi}{\ensuremath{\varphi}}
\renewcommand{\kappa}{\ensuremath{\varkappa}}
\renewcommand{\le}{\ensuremath{\leqslant}}
\renewcommand{\leq}{\ensuremath{\leqslant}}
\renewcommand{\ge}{\ensuremath{\geqslant}}
\renewcommand{\geq}{\ensuremath{\geqslant}}
\renewcommand{\emptyset}{\varnothing}

% Изменения параметров списка иллюстраций
\renewcommand{\cftfigfont}{Рисунок } % добавляем везде "Рисунок" перед номером
\addto\captionsrussian{\renewcommand\listfigurename{Список иллюстративного материала}}

\newcommand{\tm}{\texttrademark\ }
\newcommand{\reg}{\textregistered\ }




%требования к спискам, возможно пригодится
%---------------------------------------------------------
%Требования на списки в стандарте следующие:
%нумерованные списки на первом уровне помечаются как «а)», «б)», «в)»… На втором — как «1)», «2)», «3)». Да-да, я тоже не вижу тут ни капли логики.
%ненумерованные списки помечаются дефисами.

%\usepackage{enumitem}
%\makeatletter
%\AddEnumerateCounter{\asbuk}{\@asbuk}{м)}
%\makeatother
%\setlist{nolistsep}
%\renewcommand{\labelitemi}{-}
%\renewcommand{\labelenumi}{\asbuk{enumi})}
%\renewcommand{\labelenumii}{\arabic{enumii})}
%------------------------------------------------------


%%% Дополнительная работа с математикой
\usepackage{amsmath,amsfonts,amssymb,amsthm,mathtools} % AMS
\usepackage{icomma} % "Умная" запятая: $0,2$ --- число, $0, 2$ --- перечисление

%% Номера формул
%\mathtoolsset{showonlyrefs=true} % Показывать номера только у тех формул, на которые есть \eqref{} в тексте.
%\usepackage{leqno} % Нумереация формул слева

%% Свои команды
\DeclareMathOperator{\sgn}{\mathop{sgn}}

%% Перенос знаков в формулах (по Львовскому)
\newcommand*{\hm}[1]{#1\nobreak\discretionary{}
{\hbox{$\mathsurround=0pt #1$}}{}}


% отступ для первого абзаца главы или параграфа
%\usepackage{indentfirst}

%%% Работа с картинками
\usepackage{graphicx}  % Для вставки рисунков
\graphicspath{{images/}{screnshots/}}  % папки с картинками
\DeclareGraphicsExtensions{.pdf,.png,.jpg}
\setlength\fboxsep{3pt} % Отступ рамки \fbox{} от рисунка
\setlength\fboxrule{1pt} % Толщина линий рамки \fbox{}
\usepackage{wrapfig} % Обтекание рисунков текстом

%%% Работа с таблицами
\usepackage{array,tabularx,tabulary,booktabs} % Дополнительная работа с таблицами
\usepackage{longtable}  % Длинные таблицы
\usepackage{multirow} % Слияние строк в таблице

%%% Теоремы
\theoremstyle{plain} % Это стиль по умолчанию, его можно не переопределять.
\newtheorem{theorem}{Теорема}[section]
\newtheorem{proposition}[theorem]{Утверждение}

\theoremstyle{plain} % Это стиль по умолчанию, его можно не переопределять.
\newtheorem{work}{Практическая работа}[part]


 
 
\theoremstyle{definition} % "Определение"
\newtheorem{corollary}{Следствие}[theorem]
\newtheorem{problem}{Задача}[section]
 
\theoremstyle{remark} % "Примечание"
\newtheorem*{nonum}{Решение}



%%% Программирование
\usepackage{etoolbox} % логические операторы

%%% Страница

%	\usepackage{fancyhdr} % Колонтитулы
% 	\pagestyle{fancy}
%   \renewcommand{\headrulewidth}{0pt}  % Толщина линейки, отчеркивающей верхний колонтитул
% 	\lfoot{Нижний левый}
% 	\rfoot{Нижний правый}
% 	\rhead{Верхний правый}
% 	\chead{Верхний в центре}
% 	\lhead{Верхний левый}
%	\cfoot{Нижний в центре} % По умолчанию здесь номер страницы

\usepackage{setspace} % Интерлиньяж
\onehalfspacing % Интерлиньяж 1.5
%\doublespacing % Интерлиньяж 2
%\singlespacing % Интерлиньяж 1

\usepackage{lastpage} % Узнать, сколько всего страниц в документе.

\usepackage{soul} % Модификаторы начертания


\usepackage[usenames,dvipsnames,svgnames,table,rgb]{xcolor}


\usepackage{csquotes} % Еще инструменты для ссылок

%\usepackage[style=authoryear,maxcitenames=2,backend=biber,sorting=nty]{biblatex}

\usepackage{multicol} % Несколько колонок

\usepackage{tikz} % Работа с графикой
\usepackage{pgfplots}
\usepackage{pgfplotstable}

% модуль для вставки рыбы
\usepackage{blindtext}

\usepackage{listings}
\usepackage{color}


% для поворота отдельной страницы. Использовать окружение \landscape
\usepackage{pdflscape} 
\usepackage{rotating} 


\definecolor{mygreen}{rgb}{0,0.6,0}
\definecolor{mygray}{rgb}{0.5,0.5,0.5}
\definecolor{mymauve}{rgb}{0.58,0,0.82}


% пример импорта файла
%\lstinputlisting{/home/denilai/repomy/conf/distributions}

\lstset{
	language=Python,
	basicstyle=\footnotesize,        % the size of the fonts that are used for the code
	numbers=left,                    % where to put the line-numbers; possible values are (none, left, right)
	numbersep=5pt,                   % how far the line-numbers are from the code
	numberstyle=\tiny\color{mygray}, % the style that is used for the line-numbers
	stepnumber=2,                    % the step between two line-numbers. If it's 1, each line will be numbered
	% Tab - 2 пробела
	tabsize=2,    
	% Автоматический перенос строк
	breaklines=true,
	frame=single,
	breakatwhitespace=true,
	title=\lstname 
}





\author{Кирилл Денисов}
\title{Практическая работа №2}
\date{\today}

% установка полуторного интервала
% \usepackage{setspace}  
% \onehalfspacing

% использовать Times New Roman
\renewcommand{\rmdefault}{ftm}


\begin{document}
	\thispagestyle{empty}
	% Вставка первого титульного листа
	%\newcounter{withouttheme}

%\setcounter{withouttheme}{<n>} установить значение счетчика  withouttheme для определения, нужна ли тема
%    {0} - нужна
%    {1} - не нужна

%\setcounter{withoutsubmissiondate}{<n>} установить значение счетчика  withoutsubmissiondate для определения, нужна ли дата представления к защите
%     {0} - нужна
%     {1} - не нужена
\begin{center}
	\begin{figure}[h!]
		\begin{center}
		%\vspace{-10ex}
		\includegraphics[width=0.17\linewidth]{\pathToCommonFolder/gerb}
		%\caption{}\label{pic:first}
		%	\vspace{5ex}
		\end{center}	
	\end{figure}
 	\small	МИНОБРНАУКИ РОССИИ \\
	Федеральное государственное бюджетное образовательное учреждение\\
						высшего образования\\
\normalsize					
\textbf{«МИРЭА – Российский технологический университет»\\
						РТУ МИРЭА}\\
						\noindent\rule{1\linewidth}{1pt}\\
       Институт информационных технологий\\ %\vspace{2ex}
					\kafedra\\
		\vspace{3ex}
			\large \textbf{\workname}  \\
		%\vspace{1ex}
						по дисциплине\\ «\discipline» \\
		\vspace{3ex}
		\ifnum \value{withouttheme}=0 {
			\textbf{Тема работы:}\\ <<\theme>>
		}
		\else {}
		\fi
\vspace{10ex}
\small
\begin{table}[h!]
\begin{tabular}{lp{0.6\linewidth}l}
	\textbf{Выполнил:} & студент группы ИВБО-02-19 & \\ 
	& & \studentfio \\%Д.~Н.~Федосеев\\%А.~М.~Сосунов\\%К.~Ю.~Денисов\\%И.~А.~Кремнев
	\textbf{Принял:} & \rang & \\
	& & \teacherfio \hfill\\
\end{tabular}
\end{table}
\end{center}
\ifnum \value{withoutsubmissiondate}=0 {
	\begin{flushleft}
		Работа представлена к защите <<\rule{3ex}{1pt}>>\rule{10ex}{1pt} 202\rule{1ex}{1pt} г.\hfill
	\end{flushleft}
\else {}
\fi

\normalsize
\begin{center}	
\vfill
Москва 2022
\end{center}

	\newpage
	\tableofcontents
	\newpage
	
\section{Цели и задачи работы}
\textbf{Цель работы: }изучить
структуру и
информационной системы.

\textbf{Задача: }
необходимо детально описать функционал системы в соответствии с
индивидуальным вариантом учебного проекта.

\section {Индивидуальный вариант № 6}
В качестве задания было предложено изучить и смоделировать процесс организации оптового бизнеса. Собрать предварительную информацию и составить описание объекта автоматизации. Описать основные функции данной системы.
\section{Ход работы}
\subsection{Общие сведения}
Одной из основных задач торговли является обеспечение эффективного товародвижения от производителей к конечным покупателям (потребителям). Во многих случаях данные процессы не могут быть осуществлены без участия оптовой торговли, призванной обеспечить соответствующее накопление необходимых товаров и их перемещение.

Обычно под оптовой торговлей понимается любая деятельность по продаже товаров или услуг для их дальнейшей перепродажи или производственного использования.
Оптовую торговлю могут осуществлять как товаропроизводители, так и различные коммерческие оптовые фирмы. Непосредственное участие в оптовой торговле принимают агенты и брокеры.
\subsubsection{Прямая оптовая торговля}
Прямую оптовую торговлю осуществляют товаропроизводители. Это они делают тогда, когда считают, что таким способом они смогут обеспечить наиболее эффективную политику продаж.

В условиях прямой оптовой торговли товаропроизводители обычно создают торговые филиалы. Также в реализации товара могут участвовать отделы сбыта.

Торговые филиалы создают запасы товаров фирмы и обеспечивают реализацию широкого круга функций оптовой торговли.

\subsubsection{Дилеры}
В качестве дилера может быть физическое лицо или фирма, являющаяся посредником в торговых сделках купли-продажи товаров, ценных бумаг, валюты. Такой посредник действует от своего имени и за свой счет. Свои доходы дилер получает за счет более высокой цены продажи товаров по сравнению с ценой покупки.

\subsubsection{Дистрибьюторы}
Дистрибьютором считается независимая коммерческая фирма, осуществляющая свою предпринимательскую деятельность благодаря совершению оптовых закупок у производителей в целях ее перепродажи. Дистрибьюторы, как правило, устанавливают прямые длительные связи с производителями и покупателями продукции.

\subsection{Структура системы}
Рассмотрим структуру оптового предприятия для дальнейшего уточнения объектов автоматизации. См. рисунок \ref{img:struct}.

\begin{figure}[h!]
	\centering
	\includegraphics[width=0.9\linewidth]{1}
	\caption{Структура оптового предприятия}
	\label{img:struct}
\end{figure}
\newpage
\subsection{Основные функции системы}
Рассмотрим функции основных структурных отделов информационной системы.
\small

\begin{longtable}[h!]{|p{0.3\linewidth}|p{0.7\linewidth}|}
	\hline
	\textbf{Название} & \textbf{Краткое описание} \\
	\hline
	\endfirsthead
	\hline
	\textbf{Название} & \textbf{Краткое} \\
	\hline \endhead
	\hline
	\multicolumn{2}{c}{\textit{Продолжение на следующей странице}}
	\endfoot
	\hline \endlastfoot
	\hline
	\multicolumn{2}{|c|}{\textbf{Функции директора оптового предприятия}} \\ \hline
	Управляющая & Руководит в соответствии с действующим законодательством производственно-хозяйственной и финансово-экономической деятельностью предприятия, неся всю полноту ответственности за последствия принимаемых решений.Обязан обеспечить сохранность и не допусть расхищения материальных ценностей предприятия. \\ \hline
	Контролирующая & Распределяет работу среди своих прямых подчиненных: руководителей отделов, подразделений, производственных цехов и т.д.\\ \hline
	Надзорная & Следит за соблюдением норм охраны труда, создания условий отдыха для коллектива.\\ \hline
	
	\multicolumn{2}{|c|}{\textbf{Функции планово--экономического отдела}} \\ \hline
	Оценочная     & Определение уровня общественно необходимых затрат труда
	через ценообразование;\\ \hline
	Организующая и
	регулирующая  & Обеспечение рационального построения и гармонического функционирования
	экономических систем с помощью
	импульсов, стимулирующих структурные изменения.\\ \hline
	{Регулирование рынка} & Деятельность, направленная на сглаживание цен,
	преобразование промышленного ассортимента в товарный. \\\hline
	
	\multicolumn{2}{|c|}{\textbf{Функции торгового и товарного отдела}} \\ \hline
	Интегрирующая & Обеспечение взаимосвязи между партнерами по поставкам продукции
	по нахождению каналов сбыта\\ \hline
	Распределяющая & Управление и оптимизация реализации товарной массы\\ \hline
	Накопление и хранение & Накопление и хранение товарных запасов, концентрация товарной массы,
	доведение товара до требуемого качества, фасовка, упаковка\\ \hline
	
	\multicolumn{2}{|c|}{\textbf{Функции транспортного отдела}} \\ \hline
	Транспортное обеспечение деятельности предприятия & \begin{itemize}
		\item Разработка годовых, квартальных, месячных и оперативных планов-графиков транспортных перевозок на основе планов получения отгрузки готовой продукции;
		\item Разработка маршрутов движения;
		\item Передача материально-технических ресурсов на склады предприятия и передача готовой продукции на склады;
		\item Составление отчетов выполнения планов грузопереработок;
		\item Оформление транспортной документации;
		\item Содержание подвижного состава транспортных средств в технически исправном состоянии;
	\end{itemize} \\ \hline
	Совершенствование транспортного обеспечения предприятия & Разработка и выполнение мероприятий, обеспечивающих:
	\begin{itemize}
		\item сокращение простоя транспорта под грузовыми операциями;
		\item увеличение пропускной способности и рациональное использование площадок и путей подъезда ТС;
		\item рациональное использование погрузочно-разгрузочных механизмов и ТС;
		\item устранение причин преждевременных возвратов транспортных средств с линий из-за технических неисправностей.
	\end{itemize} \\ \hline

	\multicolumn{2}{|c|}{\textbf{Функции отдела маркетинга}} \\ \hline
	{Маркетинговая} & Увеличение клиентской базы за счет предоставления маркетинговых услуг \\ \hline
	Аналитическая & Анализ эффективности принятия маркетинговых решений. Изучение объемов поставки, качества конкурирующей продукции, ее преимуществ и недостатков по сравнению с продукцией данного предприятия.
	
	Осуществление непосредственных контактов с потребителями продукции;
\end{longtable}
\normalsize
\subsection{Объекты автоматизации}
В данной практической работе в качестве отделов для внедрения рационализаторского предложения и объектов автоматизации рассматриваются транспортный, маркетинговый и товарный отделы.
\small
\begin{longtable}[h!]{|p{0.3\linewidth}|p{0.7\linewidth}|}
	\hline
	\textbf{Название} & \textbf{Краткое описание} \\
	\hline \endfirsthead
	\hline
	\textbf{Название} & \textbf{Краткое описание} \\
	\hline \endhead
	\hline
	\multicolumn{2}{c}{\textit{Продолжение на следующей странице}}
	\endfoot
	\hline \endlastfoot
	\hline
	\multicolumn{2}{|c|}{\textbf{Функции транспортного отдела}} \\ \hline
	Транспортное обеспечение деятельности предприятия & Использование программного обеспечения, позволяющего построить оптимальные маршруты перевозок в режиме реального времени положительно отразится на времени доставки товаров со складов. 
	
	
	Учет системой местоположения подвижного состава, позволит минимизировать простой автомобильной техники, а использование программ, отслеживающих состояние внутренних систем, степень износа узлов и проведение своевременных профилактических мероприятий на основе полученных данных, позволят увеличить срок службы и количество моточасов между капитальным ремонтом подвижного состава.\\ \hline
	\multicolumn{2}{|c|}{\textbf{Функции товарного отдела}} \\ \hline
	Распределяющая функция & Использование базы данных для учета прихода и расхода товара, контроля за сроками поставок, хранения данных поставщиков и клиентов.  \\ \hline
	Накопление и хранение & Использование автоматизированной линии позволит ускорить технологические процессы, связанные с упаковкой, учетом и погрузкой товаров. \\ \hline
	Интегрирующая функция & Использование информационной системы контроля за документооборотом. Включение единой площадки для обеспечения взаимодействия с поставщиками товаров.\\ \hline
	\multicolumn{2}{|c|}{\textbf{Функции отдела маркетинга}} \\ \hline
	Аналитическая функция & Развитие сайта предприятия в сети Интернет. Использование инструментов веб-аналитики для получения наглядных отчетов, отслеживания источников веб-трафика, сбора и анализа данных, относящихся к области электронной коммерции --- Ecommerce.
\end{longtable}
\normalsize
\subsection{Вывод}
В ходе данной практической работы, после изучения структуры и функционала системы оптового предприятия, было предложено автоматизировать работу товарного отдела, отделов транспортировки и маркетинга. Данные изменения направлены на расширение возможностей системы, повышение качества ее работы и эффективности системы в целом.
	
\end{document}


