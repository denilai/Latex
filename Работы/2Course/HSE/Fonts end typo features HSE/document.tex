% Этот шаблон документа разработан в 2014 году
% Данилом Фёдоровых (danil@fedorovykh.ru) 
% для использования в курсе 
% <<Документы и презентации в \LaTeX>>, записанном НИУ ВШЭ
% для Coursera.org: http://coursera.org/course/latex .
% Исходная версия шаблона --- 
% https://www.writelatex.com/coursera/latex/3.2

\documentclass[a4paper,12pt]{article}

%%% Работа с русским языком
\usepackage{cmap}					% поиск в PDF
\usepackage{mathtext} 				% русские буквы в фомулах
\usepackage[T2A]{fontenc}			% кодировка
\usepackage[utf8]{inputenc}			% кодировка исходного текста
\usepackage[english,russian]{babel}	% локализация и переносы

%%% Дополнительная работа с математикой
\usepackage{amsmath,amsfonts,amssymb,amsthm,mathtools} % AMS
\usepackage{icomma} % "Умная" запятая: $0,2$ --- число, $0, 2$ --- перечисление

%% Номера формул
%\mathtoolsset{showonlyrefs=true} % Показывать номера только у тех формул, на которые есть \eqref{} в тексте.
%\usepackage{leqno} % Немуреация формул слева

%% Свои команды
\DeclareMathOperator{\sgn}{\mathop{sgn}}

%% Перенос знаков в формулах (по Львовскому)
\newcommand*{\hm}[1]{#1\nobreak\discretionary{}
{\hbox{$\mathsurround=0pt #1$}}{}}

%%% Работа с картинками
\usepackage{graphicx}  % Для вставки рисунков
\graphicspath{{images/}{images2/}}  % папки с картинками
\setlength\fboxsep{3pt} % Отступ рамки \fbox{} от рисунка
\setlength\fboxrule{1pt} % Толщина линий рамки \fbox{}
\usepackage{wrapfig} % Обтекание рисунков текстом

%%% Работа с таблицами
\usepackage{array,tabularx,tabulary,booktabs} % Дополнительная работа с таблицами
\usepackage{longtable}  % Длинные таблицы
\usepackage{multirow} % Слияние строк в таблице

%%% Теоремы
\theoremstyle{plain} % Это стиль по умолчанию, его можно не переопределять.
\newtheorem{theorem}{Теорема}[section]
\newtheorem{proposition}[theorem]{Утверждение}
 
\theoremstyle{definition} % "Определение"
\newtheorem{corollary}{Следствие}[theorem]
\newtheorem{problem}{Задача}[section]
 
\theoremstyle{remark} % "Примечание"
\newtheorem*{nonum}{Решение}

%%% Программирование
\usepackage{etoolbox} % логические операторы

%%% Страница
\usepackage{extsizes} % Возможность сделать 14-й шрифт
\usepackage{geometry} % Простой способ задавать поля
	\geometry{top=25mm}
	\geometry{bottom=35mm}
	\geometry{left=35mm}
	\geometry{right=20mm}
 %
\usepackage{fancyhdr} % Колонтитулы
 	\pagestyle{fancy}
 	\renewcommand{\headrulewidth}{0mm}  % Толщина линейки, отчеркивающей верхний колонтитул
 	\lfoot{Нижний левый}
 	\rfoot{Нижний правый}
 	\rhead{Верхний правый}
 	\chead{Верхний в центре}
 	\lhead{Верхний левый}
 	% \cfoot{Нижний в центре} % По умолчанию здесь номер страницы

\usepackage{setspace} % Интерлиньяж
%\onehalfspacing % Интерлиньяж 1.5
%\doublespacing % Интерлиньяж 2
%\singlespacing % Интерлиньяж 1

\usepackage{lastpage} % Узнать, сколько всего страниц в документе.

\usepackage{soul} % Модификаторы начертания

\usepackage{hyperref}
\usepackage[usenames,dvipsnames,svgnames,table,rgb]{xcolor}
\hypersetup{				% Гиперссылки
    unicode=true,           % русские буквы в раздела PDF
    pdftitle={Заголовок},   % Заголовок
    pdfauthor={Автор},      % Автор
    pdfsubject={Тема},      % Тема
    pdfcreator={Создатель}, % Создатель
    pdfproducer={Производитель}, % Производитель
    pdfkeywords={keyword1} {key2} {key3}, % Ключевые слова
    colorlinks=true,       	% false: ссылки в рамках; true: цветные ссылки
    linkcolor=red,          % внутренние ссылки
    citecolor=green,        % на библиографию
    filecolor=magenta,      % на файлы
    urlcolor=cyan           % на URL
}

%\renewcommand{\familydefault}{\sfdefault} % Начертание шрифта

\usepackage{multicol} % Несколько колонок

\author{\LaTeX{} в Вышке}
\title{3.2 Оформление документа в целом}
\date{\today}

\begin{document} % конец преамбулы, начало документа

\maketitle

\section{Кегль}

\begin{table}[h!]
	\caption{Размеры шрифта}
	\centering
		\begin{tabular}{|c|c|}
		\hline	\verb|\tiny|      & \tiny        крошечный \\
		\hline	\verb|\scriptsize|   & \scriptsize  очень маленький\\
			\hline \verb|\footnotesize| & \footnotesize  довольно маленький \\
			\hline \verb|\small|        &  \small        маленький \\
			\hline \verb|\normalsize|   &  \normalsize  нормальный \\
			\hline \verb|\large|        &  \large       большой \\
			\hline \verb|\Large|        &  \Large       еще больше \\[5pt]
			\hline \verb|\LARGE|        &  \LARGE       очень большой \\[5pt]
			\hline \verb|\huge|         &  \huge        огромный \\[5pt]
			\hline \verb|\Huge|         &  \Huge        громадный \\ \hline
		\end{tabular}
\end{table}

\begin{Huge}
	\textbf{Какой-нибудь \textit{обычный}  текст.}
\end{Huge}

Какой-нибудь \emph{текст \emph{с} выделением}.

\section{Титульный лист}

\newpage
\thispagestyle{empty}
\begin{center}
	\textit{Федеральное государственное автономное учреждение \\
		высшего профессионального образования}
	\vspace{0.5ex}
	
	\textbf{НАЦИОНАЛЬНЫЙ ИССЛЕДОВАТЕЛЬСКИЙ УНИВЕРСИТЕТ \\ <<ВЫСШАЯ ШКОЛА ЭКОНОМИКИ>>}
\end{center}
\vspace{13ex}
\begin{flushright}
	\noindent
	\textit{Фамилия Имя Отчество}
	\\
	\textit{студент факультета экономики \\(группа 211И)}
\end{flushright}
\begin{center}
	\vspace{13ex}
	\textbf{Р\,Е\,Ф\,Е\,Р\,А\,Т}
	\vspace{1ex}
	
	по какой-то дисциплине
	
	
	на тему:
	
	\textbf{\textit{<<Заголовок>>}}
	
	\vfill
	Москва 2014
\end{center}

\newpage

\section{Гиперссылки}

\url{http://hse.ru}

Сайт \href{http://hse.ru}{ВШЭ}

\section{Перечни}

\begin{itemize}
	\item[*] Первый пункт 
	\item 
	\begin{itemize}
		\item Вложенный список
		\item ляляля 
		\begin{enumerate}
			\item Первый пункт нумерованного списка
			\item Второй пункт
		\end{enumerate}
	\end{itemize}
	\item Второй пункт
\end{itemize}

\begin{enumerate}
	\begin{multicols}{2}
		\item $(1+x_1)(1+x_2)^2$;
		\item $\sqrt{x_1}$;
		\item $x_1+2x_2-10$;
		\item $(0{,}5x_1+x_2)^2$;
		\item $x_2$;
		\item $\sqrt{x_1}+\sqrt{2x_2}$;
		\item $\ln (1+x_1)+2\ln(1+x_2)$;
		\item $5x_1$;
		\item $10-x_1+2x_2$.
	\end{multicols}
\end{enumerate}



\end{document} % конец документа

