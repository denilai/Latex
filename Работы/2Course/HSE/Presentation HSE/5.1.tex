% Этот шаблон документа разработан в 2014 году
% Данилом Фёдоровых (danil@fedorovykh.ru) 
% для использования в курсе 
% <<Документы и презентации в \LaTeX>>, записанном НИУ ВШЭ
% для Coursera.org: http://coursera.org/course/latex .
% Исходная версия шаблона --- 
% https://www.writelatex.com/coursera/latex/5.1

\documentclass[t]{beamer}  % [t], [c], или [b] --- вертикальное выравнивание на слайдах (верх, центр, низ)
%\documentclass[handout]{beamer} % Раздаточный материал (на слайдах всё сразу)
%\documentclass[aspectratio=169]{beamer} % Соотношение сторон

%\usetheme{Berkeley} % Тема оформления
%\usetheme{Bergen}
%\usetheme{Szeged}

%\usecolortheme{beaver} % Цветовая схема
%\useinnertheme{circles}
%\useinnertheme{rectangles}

\usetheme{HSE}

%%% Работа с русским языком
\usepackage{cmap}					% поиск в PDF
\usepackage{mathtext} 				% русские буквы в формулах
\usepackage[T2A]{fontenc}			% кодировка
\usepackage[utf8]{inputenc}			% кодировка исходного текста
\usepackage[english,russian]{babel}	% локализация и переносы

%% Beamer по-русски
\newtheorem{rtheorem}{Теорема}
\newtheorem{rproof}{Доказательство}
\newtheorem{rexample}{Пример}

%%% Дополнительная работа с математикой
\usepackage{amsmath,amsfonts,amssymb,amsthm,mathtools} % AMS
\usepackage{icomma} % "Умная" запятая: $0,2$ --- число, $0, 2$ --- перечисление

%% Номера формул
%\mathtoolsset{showonlyrefs=true} % Показывать номера только у тех формул, на которые есть \eqref{} в тексте.
%\usepackage{leqno} % Нумерация формул слева

%% Свои команды
\DeclareMathOperator{\sgn}{\mathop{sgn}}

%% Перенос знаков в формулах (по Львовскому)
\newcommand*{\hm}[1]{#1\nobreak\discretionary{}
{\hbox{$\mathsurround=0pt #1$}}{}}

%%% Работа с картинками
\usepackage{graphicx}  % Для вставки рисунков
\graphicspath{{images/}{images2/}}  % папки с картинками
\setlength\fboxsep{3pt} % Отступ рамки \fbox{} от рисунка
\setlength\fboxrule{1pt} % Толщина линий рамки \fbox{}
\usepackage{wrapfig} % Обтекание рисунков текстом

%%% Работа с таблицами
\usepackage{array,tabularx,tabulary,booktabs} % Дополнительная работа с таблицами
\usepackage{longtable}  % Длинные таблицы
\usepackage{multirow} % Слияние строк в таблице

%%% Программирование
\usepackage{etoolbox} % логические операторы

%%% Другие пакеты
\usepackage{lastpage} % Узнать, сколько всего страниц в документе.
\usepackage{soul} % Модификаторы начертания
\usepackage{csquotes} % Еще инструменты для ссылок
%\usepackage[style=authoryear,maxcitenames=2,backend=biber,sorting=nty]{biblatex}
\usepackage{multicol} % Несколько колонок

%%% Картинки
\usepackage{tikz} % Работа с графикой
\usepackage{pgfplots}
\usepackage{pgfplotstable}

\title{5.1. Презентации: пакет beamer}
\subtitle{Документы и презентации в \LaTeX}
\author{Данил Фёдоровых}
\date{\today}
\institute[Высшая школа экономики]{Национальный исследовательский университет \\ <<Высшая школа экономики>>}

\begin{document}

\frame[plain]{\titlepage}	% Титульный слайд

\section{Поочередное появление объектов}
\subsection{Команда pause}
 
\begin{frame}
	\frametitle{\insertsection} 
	\framesubtitle{\insertsubsection}
	\begin{itemize}
		\item beamer ---  это \alert{удобный} \textbf{пакет} для создания презентаций.
		\item Это наш первый слайд \pause
		\item Вот полное руководство по beamer:  \href{http://ctan.uni-altai.ru/macros/latex/contrib/beamer/doc/beameruserguide.pdf}{http://ctan.uni-altai.ru/macros/latex/contrib/beamer/doc/beameruserguide.pdf} \pause
		\item Паузу можно поставить в любом \pause месте.
		\item Для печати презентации есть режим handout.
	\end{itemize}
\end{frame}

\subsection{Более сложно}

\begin{frame}
	\frametitle{Поочередное появление пунктов списка}
	\framesubtitle{\insertsubsection}
    \uncover<4->{Эта строчка появляется не сразу, но занимает место.}
    \only<5->{Эта строчка появляется не сразу и не занимает места.}
    \begin{enumerate}
        \item<1-5> Сначала появляются первый и последний пункт (первый потом исчезнет).
        \item<2-> Потом второй
        \item<3-> И наконец третий
        \item<1-> Последний пункт появляется вместе с первым
    	\item<6-> В самом конце первый пункт исчезает, зато появляется картинка: \insertlogo.
    \end{enumerate}
    \alt<4>{Это \alert{четвертый} слайд}{Это не четвертый слайд.}
    \temporal<3-4>{Слайды 1, 2}{Слайды 3, 4}{Сайды 5, 6, 7, ...}
\end{frame}

\begin{frame}{Пример из руководства}
   \textbf{This line is bold on all three slides.}
   \textbf<2>{This line is bold only on the second slide.}
   \textbf<3>{This line is bold only on the third slide.}
   \textbf<3,4>{Эта строчка полужирная на 3-м и 4-м слайде.}
	\color<3-4>[RGB]{255,0,0} Этот текст красный только на сладах 3-4.
\end{frame}

\section{Об оформлении}
\subsection{Презентация в целом}

\begin{frame}[c] % [t], [c], или [b] --- вертикальное выравнивание на слайде (верх, центр, низ)
	\frametitle{\insertsection}
	\framesubtitle{\insertsubsection}
	\begin{itemize}
		\item Варианты оформления можно посмотреть здесь:  \href{http://www.hartwork.org/beamer-theme-matrix/}{http://www.hartwork.org/beamer-theme-matrix/} 
		\item Символика НИУ ВШЭ в формате PDF: \href{http://www.hse.ru/org/hse/info/logo}{http://www.hse.ru/org/hse/info/logo}
		\item Можно создавать свои темы.
		\item Можно выровнять текст на слайдах по верхнему краю с использованием опции [t] (у всей презентации ---  \texttt{[t]{beamer}} или у отдельного слайда \texttt{{frame}[t]}) 
		\item Можно регулировать формат слайдов с помощью опции \texttt{[aspectratio=xx]}. Например, \texttt{[aspectratio=169]}. 
	\end{itemize}
\end{frame}

\subsection{На слайде}

\begin{frame}
	\frametitle{\insertsection}
	\framesubtitle{\insertsubsection}
    \begin{block}{Первый блок}
		Текст первого блока
	\end{block}
	\begin{block}{Второй блок}
		Текст второго блока
    \end{block}
	\begin{block}{Кнопка}
		\hyperlink{lab}{\beamerbutton{Кнопка со ссылкой}} 
	\end{block}
\end{frame}

\begin{frame}
	\frametitle{\insertsection}
	\framesubtitle{\insertsubsection}
    \begin{rtheorem}
		Формулировка теоремы.    
	\end{rtheorem}
	\begin{rproof}
		Текст доказательства.
    \end{rproof}
	\begin{rexample}
		Текст примера.
	\end{rexample}
\end{frame}

\section{Формулы}

\begin{frame}
	\frametitle{\insertsection}
	\begin{itemize}
		\item $\displaystyle\int_0^1 x=1/2$  
		\item $2\times 2=4$, $2^2=4$ 
		\item $\displaystyle\frac{5}{\alert{6}}=\frac{1}{1+\frac{1}{5}}$ 
	\end{itemize}
\end{frame}

\section{Большой текст}

\begin{frame}[shrink=7]\label{lab}  % Опцию shrink лучше не использовать
	\frametitle{\insertsection}
	\textsl{Чуден Днепр при тихой погоде, когда вольно и плавно мчит сквозь леса и горы полные воды свои. Ни зашелохнет; ни прогремит. Глядишь, и не знаешь, идет или не идет его величавая ширина, и чудится, будто весь вылит он из стекла, и будто голубая зеркальная дорога, без меры в ширину, без конца в длину, реет и вьется по зеленому миру. Любо тогда и жаркому солнцу оглядеться с вышины и погрузить лучи в холод стеклянных вод и прибережным лесам ярко отсветиться в водах. Зеленокудрые! они толпятся вместе с полевыми цветами к водам и, наклонившись, глядят в них и не наглядятся, и не налюбуются светлым  своим зраком, и усмехаются к нему, и приветствуют его, кивая ветвями. В середину же Днепра они не смеют глянуть: никто, кроме солнца и голубого неба, не глядит в него. Редкая птица долетит до середины Днепра. Пышный! ему нет равной реки в мире. Чуден Днепр и при теплой летней ночи, когда все засыпает --- и человек, и зверь, и птица; а бог один величаво озирает небо и землю и величаво сотрясает ризу.}

	\hfill{\textit{Н. В. Гоголь}}
\end{frame}
 
\includeonlylecture{lec1} % Можно поместить это в преамбулу

\lecture{Лекция 1}{lec1}
	\section{<<Лекции>>}
	\begin{frame}
		\frametitle{\insertsection}
		lecture-функционал позволяет включать в презентацию только отдельные слайды, а в одном файле держать целый цикл презентаций 
	\end{frame}

\lecture{Лекция 2}{lec2}
	\section{Это лекция 2}
	\begin{frame}
	\frametitle{\insertsection}
		Все frames не внутри какой-нибудь lecture присутствуют всегда. Если нет  команды \verb"includeonlylecture{}", то все лекции тоже присутствуют всегда. Отдельную лекцию можно показать с помощью \verb"includeonlylecture{}".
	\end{frame}

\end{document}