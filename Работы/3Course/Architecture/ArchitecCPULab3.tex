\documentclass[a4paper,14pt]{extarticle}

% Путь до папки с общими шаблонами
\newcommand{\pathToCommonFolder}{/home/denilai/Documents/repos/latex/Common}
% Название работы в титуле
\newcommand{\workname}{Отчет по лабораторной работе №3}
% Название дисциплины в титуле
\newcommand{\discipline}{Архитектура процессоров и микропроцессоров}
% Название кафедры в титуле
\newcommand{\kafedra}{Кафедра вычислительной техники}
% Тема работы в титуле
\newcommand{\theme}{Стадии выполнения команд процессором КР580ВМ80}
% Должность преподавателя в титуле
\newcommand{\rang}{cтарший преподаватель кафедры ВТ}
% ФИО преподавателя в титуле
\newcommand{\teacherfio}{Ю.~М.Скрябин}
\newcommand{\studentfio}{К.~Ю.~Денисов}%Д.~Н.~Федосеев\\%А.~М.~Сосунов\\%К.~Ю.~Денисов\\

\usepackage{tabularx}



\usepackage{booktabs}
\newcolumntype{b}{X}
\newcolumntype{s}{>{\hsize=.5\hsize}X}
\newcommand{\heading}[1]{\multicolumn{1}{c}{#1}}

% установка размера шрифта для всего документа
%\fontsize{20pt}{18pt}\selectfont
\usepackage{extsizes} % Возможность сделать 14-й шрифт

% Вставка заготовки преамбулы
% Этот шаблон документа разработан в 2014 году
% Данилом Фёдоровых (danil@fedorovykh.ru) 
% для использования в курсе 
% <<Документы и презентации в \LaTeX>>, записанном НИУ ВШЭ
% для Coursera.org: http://coursera.org/course/latex .
% Исходная версия шаблона --- 
% https://www.writelatex.com/coursera/latex/5.3

% В этом документе преамбула

% Для корректного использования русских символов в формулах
% пакеты hyperref и настройки, связанные с ним, стоит загуржать
% перед загрузкой пакета mathtext



% поддержка русских букв
% кодировка шрифта
%\usepackage[T2A]{fontenc} 
\usepackage{pscyr}

% использование ненумеровонного абзаца с добавлением его в содержаниеl

\newcommand{\anonsection}[1]{\section*{#1}\addcontentsline{toc}{section}{#1}}
\newcommand{\sectionunderl}[1]{\section*{\underline{#1}}}


% настройка окружения enumerate
\usepackage{enumitem}
\setlist{noitemsep}
\setlist[enumerate]{labelsep=*, leftmargin=1.5pc}

\usepackage{hyperref}

% сначала ставить \usepackage{extsizes} % Возможность сделать 14-й шрифт
% для корректной установки полей вставлять преамбулу следует в последнюю очередь (но перед дерективой замены \rmdefault)
\usepackage[top=20mm,bottom=25mm,left=35mm,right=15mm]{geometry} % Простой способ задавать поля

\hypersetup{				% Гиперссылки
	unicode=true,           % русские буквы в раздела PDF
	pdftitle={Заголовок},   % Заголовок
	pdfauthor={Автор},      % Автор
	pdfsubject={Тема},      % Тема
	pdfcreator={Создатель}, % Создатель
	pdfproducer={Производитель}, % Производитель
	pdfkeywords={keyword1} {key2} {key3}, % Ключевые слова
	colorlinks=true,       	% false: ссылки в рамках; true: цветные ссылки
	linkcolor=red,          % внутренние ссылки
	citecolor=black,        % на библиографию
	filecolor=magenta,      % на файлы
	urlcolor=blue           % на URL
}

%%% Работа с русским языком
\usepackage{cmap}					% поиск в PDF
\usepackage{mathtext} 				% русские буквы в формулах
\usepackage[T2A]{fontenc}			% кодировка
\usepackage[utf8]{inputenc}			% кодировка исходного текста
\usepackage[english,russian]{babel}	% локализация и переносы
\usepackage{indentfirst}
\frenchspacing

%для изменения названия списка иллюстраций
\usepackage{tocloft}


\renewcommand{\epsilon}{\ensuremath{\varepsilon}}
\renewcommand{\phi}{\ensuremath{\varphi}}
\renewcommand{\kappa}{\ensuremath{\varkappa}}
\renewcommand{\le}{\ensuremath{\leqslant}}
\renewcommand{\leq}{\ensuremath{\leqslant}}
\renewcommand{\ge}{\ensuremath{\geqslant}}
\renewcommand{\geq}{\ensuremath{\geqslant}}
\renewcommand{\emptyset}{\varnothing}

% Изменения параметров списка иллюстраций
\renewcommand{\cftfigfont}{Рисунок } % добавляем везде "Рисунок" перед номером
\addto\captionsrussian{\renewcommand\listfigurename{Список иллюстративного материала}}

\newcommand{\tm}{\texttrademark\ }
\newcommand{\reg}{\textregistered\ }




%требования к спискам, возможно пригодится
%---------------------------------------------------------
%Требования на списки в стандарте следующие:
%нумерованные списки на первом уровне помечаются как «а)», «б)», «в)»… На втором — как «1)», «2)», «3)». Да-да, я тоже не вижу тут ни капли логики.
%ненумерованные списки помечаются дефисами.

%\usepackage{enumitem}
%\makeatletter
%\AddEnumerateCounter{\asbuk}{\@asbuk}{м)}
%\makeatother
%\setlist{nolistsep}
%\renewcommand{\labelitemi}{-}
%\renewcommand{\labelenumi}{\asbuk{enumi})}
%\renewcommand{\labelenumii}{\arabic{enumii})}
%------------------------------------------------------


%%% Дополнительная работа с математикой
\usepackage{amsmath,amsfonts,amssymb,amsthm,mathtools} % AMS
\usepackage{icomma} % "Умная" запятая: $0,2$ --- число, $0, 2$ --- перечисление

%% Номера формул
%\mathtoolsset{showonlyrefs=true} % Показывать номера только у тех формул, на которые есть \eqref{} в тексте.
%\usepackage{leqno} % Нумереация формул слева

%% Свои команды
\DeclareMathOperator{\sgn}{\mathop{sgn}}

%% Перенос знаков в формулах (по Львовскому)
\newcommand*{\hm}[1]{#1\nobreak\discretionary{}
{\hbox{$\mathsurround=0pt #1$}}{}}


% отступ для первого абзаца главы или параграфа
%\usepackage{indentfirst}

%%% Работа с картинками
\usepackage{graphicx}  % Для вставки рисунков
\graphicspath{{images/}{screnshots/}}  % папки с картинками
\DeclareGraphicsExtensions{.pdf,.png,.jpg}
\setlength\fboxsep{3pt} % Отступ рамки \fbox{} от рисунка
\setlength\fboxrule{1pt} % Толщина линий рамки \fbox{}
\usepackage{wrapfig} % Обтекание рисунков текстом

%%% Работа с таблицами
\usepackage{array,tabularx,tabulary,booktabs} % Дополнительная работа с таблицами
\usepackage{longtable}  % Длинные таблицы
\usepackage{multirow} % Слияние строк в таблице

%%% Теоремы
\theoremstyle{plain} % Это стиль по умолчанию, его можно не переопределять.
\newtheorem{theorem}{Теорема}[section]
\newtheorem{proposition}[theorem]{Утверждение}

\theoremstyle{plain} % Это стиль по умолчанию, его можно не переопределять.
\newtheorem{work}{Практическая работа}[part]


 
 
\theoremstyle{definition} % "Определение"
\newtheorem{corollary}{Следствие}[theorem]
\newtheorem{problem}{Задача}[section]
 
\theoremstyle{remark} % "Примечание"
\newtheorem*{nonum}{Решение}



%%% Программирование
\usepackage{etoolbox} % логические операторы

%%% Страница

%	\usepackage{fancyhdr} % Колонтитулы
% 	\pagestyle{fancy}
%   \renewcommand{\headrulewidth}{0pt}  % Толщина линейки, отчеркивающей верхний колонтитул
% 	\lfoot{Нижний левый}
% 	\rfoot{Нижний правый}
% 	\rhead{Верхний правый}
% 	\chead{Верхний в центре}
% 	\lhead{Верхний левый}
%	\cfoot{Нижний в центре} % По умолчанию здесь номер страницы

\usepackage{setspace} % Интерлиньяж
\onehalfspacing % Интерлиньяж 1.5
%\doublespacing % Интерлиньяж 2
%\singlespacing % Интерлиньяж 1

\usepackage{lastpage} % Узнать, сколько всего страниц в документе.

\usepackage{soul} % Модификаторы начертания


\usepackage[usenames,dvipsnames,svgnames,table,rgb]{xcolor}


\usepackage{csquotes} % Еще инструменты для ссылок

%\usepackage[style=authoryear,maxcitenames=2,backend=biber,sorting=nty]{biblatex}

\usepackage{multicol} % Несколько колонок

\usepackage{tikz} % Работа с графикой
\usepackage{pgfplots}
\usepackage{pgfplotstable}

% модуль для вставки рыбы
\usepackage{blindtext}

\usepackage{listings}
\usepackage{color}


% для поворота отдельной страницы. Использовать окружение \landscape
\usepackage{pdflscape} 
\usepackage{rotating} 


\definecolor{mygreen}{rgb}{0,0.6,0}
\definecolor{mygray}{rgb}{0.5,0.5,0.5}
\definecolor{mymauve}{rgb}{0.58,0,0.82}


% пример импорта файла
%\lstinputlisting{/home/denilai/repomy/conf/distributions}

\lstset{
	language=Python,
	basicstyle=\footnotesize,        % the size of the fonts that are used for the code
	numbers=left,                    % where to put the line-numbers; possible values are (none, left, right)
	numbersep=5pt,                   % how far the line-numbers are from the code
	numberstyle=\tiny\color{mygray}, % the style that is used for the line-numbers
	stepnumber=2,                    % the step between two line-numbers. If it's 1, each line will be numbered
	% Tab - 2 пробела
	tabsize=2,    
	% Автоматический перенос строк
	breaklines=true,
	frame=single,
	breakatwhitespace=true,
	title=\lstname 
}



\author{Кирилл Денисов}
\title{Лабораторная работа №1}
\date{\today}

% установка полуторного интервала
% \usepackage{setspace}  
% \onehalfspacing

% использовать Times New Roman
\renewcommand{\rmdefault}{ftm}
\renewcommand{\withouttheme}{1}

\begin{document}
	\thispagestyle{empty}
	% Вставка первого титульного листа
	%\newcommand{\withouttheme}{} добавить эту переменную для определения, нужна ли тема
%     {} - нужна
%    {1} - не нужна

%\newcommand{\withoutsubmissiondate}{} добавить эту переменную для определения, нужен ли срок предоставления отчета
%     {} - нужен
%    {1} - не нужен
\begin{center}
	\begin{figure}[h!]
		\begin{center}
		\includegraphics[width=0.17\linewidth]{\pathToCommonFolder/gerb}
		%\caption{}\label{pic:first}
		%	\vspace{5ex}
		\end{center}	
	\end{figure}
 	\small	МИНОБРНАУКИ РОССИИ \\
	Федеральное государственное бюджетное образовательное учреждение\\
						высшего профессионального образования\\
\normalsize					
\textbf{«МИРЭА – Российский технологический университет»\\
						РТУ МИРЭА}\\
						\noindent\rule{1\linewidth}{1pt}\\
       Институт информационных технологий\\ %\vspace{2ex}
					\kafedra\\
		\vspace{3ex}
			\large \textbf{\workname}  \\
		%\vspace{1ex}
						по дисциплине\\ «\discipline» \\
		\vspace{3ex}
		\if \withouttheme
			\textbf{Тема работы:}\\ <<\theme>>
		\fi
\vspace{3ex}
\small
\begin{table}[h!]
\begin{tabular}{lp{0.38\linewidth}p{0.2\linewidth}p{0.2\linewidth}}
	\textbf{Выполнил:} & студент группы ИВБО-02-19 & \studentfio &\includegraphics[width=\linewidth]{\pathToCommonFolder/signature}\\ \\
	\textbf{Принял:} & \rang & \teacherfio 
\end{tabular}
\end{table}
\end{center}

\begin{flushleft}
	\begin{tabular}{p{0.25\linewidth}l}

		Работа выполена & <<\noindent\rule{2em}{1pt}>>
		                    \noindent\rule{5em}{1pt} 202\noindent\rule{1em}{1pt} \\

		<<Зачтено>> & <<\noindent\rule{2em}{1pt}>>
		\noindent\rule{5em}{1pt} 202\noindent\rule{1em}{1pt} \\

	\end{tabular}
\end{flushleft}

\normalsize
\begin{center}	
\vfill 
Москва 2021
\end{center}

	\newpage
%	\tableofcontents
%	\newpage
	%\listoftables
	

	
\section{Цель работы}
Изучить структуру эмулятора, систему команд, режимы работы. Для
программы построить временную диаграмму работы
конвейера. Пояснить, что происходит в конвейере в каждом
такте, какие возникают конфликты, указать причину конфликта.
\section {Индивидуальный вариант № 9}

\begin{problem}
	Изучить работу команд условных переходов данной программы:
	
	0000 MOV 00, 0003
	
	0001 DECR 00
	
	0002 JP 0001
	
	0003 JMP 0001
	
	\nonum{} Приведем временную диаграмму (см. таблицу~\ref{tab:time-1}).
\end{problem}


\begin{problem}
	Разработать программу для вычисления суммы первых десяти членов натурального ряда ($n=10$),
	ввести в эмулятор, исследовать ее выполнение, выявить конфликты по
	данным. Построить временную диаграмму работы конвейера. Пояснить
	возникающие конфликты, указав № такта.

	\nonum Опишем программу, реализующую алгоритм нахождения суммы членов натурального ряда (см. таблицу~\ref{tab:prog-2}).
	\newpage
	\begin{table}[htbp]

		\begin{tabular}{|l|p{0.7\linewidth}|}
			\hline
			\centerboldcell{Команда} & \centerboldcell{Описание} \\ \hline\hline
			MOV 00, \#000A & Запись в РОН 00 числа 10 \\ \hline
			ADD 01, 00, 01 & К РОН 01 прибавляем содержимое регистра 00 \\ \hline
			DECR 00 & Вычитаем 1 из РОН 00 \\ \hline
			JP 0001 & Если содеримое РОН 00 положительное, то повторяем цикл \\ \hline
		\end{tabular}
		\caption{Программа для нахождения суммы ряда}
		\label{tab:prog-2}
	\end{table}


\textbf{Описание алгоритма:}
\begin{enumerate}
	\item В РОН 00 записываем длину арифметической последовательности, т.е. 10.
	\item К РОН-аккумулятору, в котором будет накапливаться сумма последовательности прибавляем текущее значение РОН 00
	\item Уменьшаем значение РОН текущего индекса в арифметической последовательности на 1.
	\item Если результат положительный, повторяем цикл, иначе конец алгоритма.
\end{enumerate}

Построим временную диаграмму данной программы (см. таблицу~\ref{tab:time-2}).

\textbf{Опишем конфликты, возникающие при выполнении данной программы конвейером:}
\begin{enumerate}
	\item На 4 такте мы наблюдаем \textit{структурный конфликт}, так как команда ADD использует тот же РОН что и команда MOV, но команда MOV ещё не закончила своё выполнение, поэтому мы  не можем обратиться к одному и тому же РОН и для чтения и для записи
	

	\item На 9 такте мы наблюдаем\textit{ конфликт по данным}, так как команды ADD и DECR используют один и тот же операнд из РОН 00, но команада DECR ещё не закончила своё выполнение.

	
\end{enumerate}

\textbf{Опишем варианты избежания конфликтов:}
\begin{enumerate}
	\item Можно избежать конфликта, на такте 4 если поменять местами команды, чтобы чтение происходило в другом такте, но в данном случае это невозможно из-за небольшого количества команд в программе.
	
	\item Можно избежать конфликта на такте 9, если использовать обходную цепь, но так как между командами есть ещё команда JP, обходную цепь использовать не представляется возможности.
\end{enumerate}


\end{problem}



\begin{problem}
	Разработать программу для организации инкремента содержимого
	регистра РОН от 0 до 10.
	\nonum
	Опишем программу, реализующую алгоритм организации инкремента содержимого
	регистра РОН от 0 до 10. (см. таблицу~\ref{tab:prog-3}).	
	
	\begin{table}[htbp]
		\begin{tabular}{|l|p{0.7\linewidth}|}
			\hline
			\centerboldcell{Команда} & \centerboldcell{Описание} \\ \hline\hline
			MOV 00, \#000A & Записываем значение 10 в регистр 00 \\ \hline
			INCR 01 & Инкрементируем значение регистра 01 \\ \hline
			SUB 02, 01, 00 & Вычитаем из значения РОН 00 значение РОН 01 и результат записываем в РОН 02 \\ \hline
			JP 0001 & Если результат предыдущей операции положительный, то переходим на первый шаг \\ \hline
		\end{tabular}
		\caption{Программа для инкрементирования значения регистра}
		\label{tab:prog-3}
	\end{table}

\textbf{Описание алгоритма:}
\begin{enumerate}
	\item В РОН 00 записываем значение до которого происходит инкрементирование переменной, т.е. 10. Сама переменная будет находиться в РОН 01.
	\item На каждой итерации цикла увеличиваем значение РОН переменной на один.
	\item Вычитаем это значение из 10.
	\item Если результат положительный, повторяем цикл, иначе конец алгоритма.
\end{enumerate}

Построим временную диаграмму данной программы (см. таблицу~\ref{tab:time-3}).

\textbf{Опишем конфликты, возникающие при выполнении данной программы конвейером:}

На 4 такте мы наблюдаем\textit{ структурный конфликт}, так как команда INCR использует тот же РОН что и команда MOV, но команда MOV ещё не закончила своё выполнение, поэтому мы  не можем обратиться к одному и тому же РОН и для чтения и для записи.

\textbf{Опишем варианты избежания конфликта:}

 Можно избежать конфликта, на такте 4 если поменять местами команды, чтобы чтение происходило в другом такте, но в данном случае это невозможно из-за небольшого количества команд в программе.

\end{problem}	

\begin{landscape}

\begin{table}[htbp]
	\small
	\begin{tabular}{|r|l|l||l|l|l|l|l|l|l|l|l|l|l|}
		\hline
		\multicolumn{1}{|l|}{Ст/Т} & \multicolumn{1}{r|}{1} & \multicolumn{1}{r|}{2} & \multicolumn{1}{r|}{3} & \multicolumn{1}{r|}{4} & \multicolumn{1}{r|}{5} & \multicolumn{1}{r|}{6} & \multicolumn{1}{r|}{7} & \multicolumn{1}{r|}{8} & \multicolumn{1}{r|}{9} & \multicolumn{1}{r|}{10} & \multicolumn{1}{r|}{11} & \multicolumn{1}{r|}{12} & \multicolumn{1}{r|}{13} \\ \hline
		1 & MOV & DECR & JP & JP & JP & DECR & JP & JP & DECR & JP & JP & DECR & JP \\ \hline
		2 &  & MOV & DECR & DECR & DECR & JP & DECR & DECR & JP & DECR & DECR & JP & DECR \\ \hline
		3 &  &  & MOV &  &  & DECR & JP &  & DECR & JP &  & DECR & JP \\ \hline
		4 &  &  &  & MOV &  &  & DECR & JP &  & DECR & JP &  & DECR \\ \hline
		5 &  &  &  &  & MOV &  &  & DECR & JP &  & DECR & JP &  \\ \hline\hline
		\multicolumn{1}{|l|}{Ст/Т} & \multicolumn{1}{r|}{14} & \multicolumn{1}{r||}{15} & \multicolumn{1}{r|}{16} & \multicolumn{1}{r|}{17} & \multicolumn{1}{r|}{18} & \multicolumn{1}{r|}{19} & \multicolumn{1}{r|}{20} & \multicolumn{1}{r|}{21} & \multicolumn{1}{r|}{22} & \multicolumn{1}{r|}{23} & \multicolumn{1}{r|}{24} & \multicolumn{1}{r|}{25} & \multicolumn{1}{r|}{26} \\ \hline
		1 &  & JMP & DECR & JP & JMP & DECR &  & DECR & JP & DECR & JP & JP & DECR \\ \hline
		2 &  &  & JMP & DECR & JP & JMP &  &  & DECR & JP & DECR & DECR & JP \\ \hline
		3 &  &  &  & JMP & DECR & JP &  &  &  & DECR & JP &  & DECR \\ \hline
		4 & JP &  &  &  & JMP & DECR & JP &  &  &  & DECR & JP &  \\ \hline
		5 & DECR & JP &  &  &  & JMP & DECR & JP &  &  &  & DECR & JP \\ \hline
	\end{tabular}
	\caption{Задние 1. Временная диаграмма}
	\label{tab:time-1}
\end{table}
\newpage

\begin{table}[htbp]
	\begin{tabular}{|r|l|l|l|l|l|l|l|l|l|l|l|l|}
		\hline
		\multicolumn{1}{|l|}{Ст/Т} & \multicolumn{1}{r|}{1} & \multicolumn{1}{r|}{2} & \multicolumn{1}{r|}{3} & \multicolumn{1}{r|}{4} & \multicolumn{1}{r|}{5} & \multicolumn{1}{r|}{6} & \multicolumn{1}{r|}{7} & \multicolumn{1}{r|}{8} & \multicolumn{1}{r|}{9} & \multicolumn{1}{r|}{10} & \multicolumn{1}{r|}{11} & \multicolumn{1}{r|}{12} \\ \hline
		1 & MOV & ADD & DECR & DECR & DECR & JP & ADD & DECR & DECR & JP & ADD & DECR \\ \hline
		2 &  & MOV & ADD & ADD & ADD & DECR & JP & ADD & ADD & DECR & JP & ADD \\ \hline
		3 &  &  & MOV &  &  & ADD & DECR & JP &  & ADD & DECR & JP \\ \hline
		4 &  &  &  & MOV &  &  & ADD & DECR & JP &  & ADD & DECR \\ \hline
		5 &  &  &  &  & MOV &  &  & ADD & DECR & JP &  & ADD \\ \hline
	\end{tabular}
	\caption{Задание 2. Временная диаграмма}
	\label{tab:time-2}
\end{table}

\begin{table}[htbp]
	\begin{tabular}{|r|l|l|l|l|l|l|l|l|l|l|l|}
		\hline
		\multicolumn{1}{|l|}{Ст/Т} & \multicolumn{1}{r|}{1} & \multicolumn{1}{r|}{2} & \multicolumn{1}{r|}{3} & \multicolumn{1}{r|}{4} & \multicolumn{1}{r|}{5} & \multicolumn{1}{r|}{6} & \multicolumn{1}{r|}{7} & \multicolumn{1}{r|}{8} & \multicolumn{1}{r|}{9} & \multicolumn{1}{r|}{10} & \multicolumn{1}{r|}{11} \\ \hline
		1 & MOV & INCR & SUB & SUB & SUB & JP & INCR & SUB & JP & INCR & SUB \\ \hline
		2 &  & MOV & INCR & INCR & INCR & SUB & JP & INCR & SUB & JP & INCR \\ \hline
		3 &  &  & MOV &  &  & INCR & SUB & JP & INCR & SUB & JP \\ \hline
		4 &  &  &  & MOV &  &  & INCR & SUB & JP & INCR & SUB \\ \hline
		5 &  &  &  &  & MOV &  &  & INCR & SUB & JP & INCR \\ \hline
	\end{tabular}
	\caption{Задание 3. Временная диаграмма}
	\label{tab:time-3}
\end{table}

\end{landscape}


\anonsection{Вывод}

В ходе данной лабораторной работы мы ознакомились со структурой эмулятора RISC конвейера, изучили его систему команд, режимы работы, описали алгоритмы и реализовали программы согласно варианту, построили временные диаграммы работы конвейера, идентифицировали конфликты и указали способы их устранения.

\end{document}


