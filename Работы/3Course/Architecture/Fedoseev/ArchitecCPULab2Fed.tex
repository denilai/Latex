\documentclass[a4paper,14pt]{extarticle}

% Путь до папки с общими шаблонами
\newcommand{\pathToCommonFolder}{/home/denilai/Documents/repos/latex/Common}
% Название работы в титуле
\newcommand{\workname}{Отчет по лабораторной работе №2}
% Название дисциплины в титуле
\newcommand{\discipline}{Архитектура процессоров и микропроцессоров}
% Название кафедры в титуле
\newcommand{\kafedra}{Кафедра вычислительной техники}
% Тема работы в титуле
\newcommand{\theme}{Стадии выполнения команд процессором КР580ВМ80}
% Должность преподавателя в титуле
\newcommand{\rang}{cтарший преподаватель кафедры ВТ}
% ФИО преподавателя в титуле
\newcommand{\teacherfio}{Ю.~М.Скрябин}
\newcommand{\studentfio}{Д.~Н.~Федосеев}%Д.~Н.~Федосеев\\%А.~М.~Сосунов\\%К.~Ю.~Денисов\\
\newcommand{\signature}{\pathToCommonFolder/fedoseev-signature}
\def\myformat#1{\hfil #1\hfil}



\newcommand{\heading}[1]{\multicolumn{1}{c}{#1}}

% установка размера шрифта для всего документа
%\fontsize{20pt}{18pt}\selectfont
\usepackage{extsizes} % Возможность сделать 14-й шрифт

% Вставка заготовки преамбулы
% Этот шаблон документа разработан в 2014 году
% Данилом Фёдоровых (danil@fedorovykh.ru) 
% для использования в курсе 
% <<Документы и презентации в \LaTeX>>, записанном НИУ ВШЭ
% для Coursera.org: http://coursera.org/course/latex .
% Исходная версия шаблона --- 
% https://www.writelatex.com/coursera/latex/5.3

% В этом документе преамбула

% Для корректного использования русских символов в формулах
% пакеты hyperref и настройки, связанные с ним, стоит загуржать
% перед загрузкой пакета mathtext



% поддержка русских букв
% кодировка шрифта
%\usepackage[T2A]{fontenc} 
\usepackage{pscyr}

% использование ненумеровонного абзаца с добавлением его в содержаниеl

\newcommand{\anonsection}[1]{\section*{#1}\addcontentsline{toc}{section}{#1}}
\newcommand{\sectionunderl}[1]{\section*{\underline{#1}}}


% настройка окружения enumerate
\usepackage{enumitem}
\setlist{noitemsep}
\setlist[enumerate]{labelsep=*, leftmargin=1.5pc}

\usepackage{hyperref}

% сначала ставить \usepackage{extsizes} % Возможность сделать 14-й шрифт
% для корректной установки полей вставлять преамбулу следует в последнюю очередь (но перед дерективой замены \rmdefault)
\usepackage[top=20mm,bottom=25mm,left=35mm,right=15mm]{geometry} % Простой способ задавать поля

\hypersetup{				% Гиперссылки
	unicode=true,           % русские буквы в раздела PDF
	pdftitle={Заголовок},   % Заголовок
	pdfauthor={Автор},      % Автор
	pdfsubject={Тема},      % Тема
	pdfcreator={Создатель}, % Создатель
	pdfproducer={Производитель}, % Производитель
	pdfkeywords={keyword1} {key2} {key3}, % Ключевые слова
	colorlinks=true,       	% false: ссылки в рамках; true: цветные ссылки
	linkcolor=red,          % внутренние ссылки
	citecolor=black,        % на библиографию
	filecolor=magenta,      % на файлы
	urlcolor=blue           % на URL
}

%%% Работа с русским языком
\usepackage{cmap}					% поиск в PDF
\usepackage{mathtext} 				% русские буквы в формулах
\usepackage[T2A]{fontenc}			% кодировка
\usepackage[utf8]{inputenc}			% кодировка исходного текста
\usepackage[english,russian]{babel}	% локализация и переносы
\usepackage{indentfirst}
\frenchspacing

%для изменения названия списка иллюстраций
\usepackage{tocloft}


\renewcommand{\epsilon}{\ensuremath{\varepsilon}}
\renewcommand{\phi}{\ensuremath{\varphi}}
\renewcommand{\kappa}{\ensuremath{\varkappa}}
\renewcommand{\le}{\ensuremath{\leqslant}}
\renewcommand{\leq}{\ensuremath{\leqslant}}
\renewcommand{\ge}{\ensuremath{\geqslant}}
\renewcommand{\geq}{\ensuremath{\geqslant}}
\renewcommand{\emptyset}{\varnothing}

% Изменения параметров списка иллюстраций
\renewcommand{\cftfigfont}{Рисунок } % добавляем везде "Рисунок" перед номером
\addto\captionsrussian{\renewcommand\listfigurename{Список иллюстративного материала}}

\newcommand{\tm}{\texttrademark\ }
\newcommand{\reg}{\textregistered\ }




%требования к спискам, возможно пригодится
%---------------------------------------------------------
%Требования на списки в стандарте следующие:
%нумерованные списки на первом уровне помечаются как «а)», «б)», «в)»… На втором — как «1)», «2)», «3)». Да-да, я тоже не вижу тут ни капли логики.
%ненумерованные списки помечаются дефисами.

%\usepackage{enumitem}
%\makeatletter
%\AddEnumerateCounter{\asbuk}{\@asbuk}{м)}
%\makeatother
%\setlist{nolistsep}
%\renewcommand{\labelitemi}{-}
%\renewcommand{\labelenumi}{\asbuk{enumi})}
%\renewcommand{\labelenumii}{\arabic{enumii})}
%------------------------------------------------------


%%% Дополнительная работа с математикой
\usepackage{amsmath,amsfonts,amssymb,amsthm,mathtools} % AMS
\usepackage{icomma} % "Умная" запятая: $0,2$ --- число, $0, 2$ --- перечисление

%% Номера формул
%\mathtoolsset{showonlyrefs=true} % Показывать номера только у тех формул, на которые есть \eqref{} в тексте.
%\usepackage{leqno} % Нумереация формул слева

%% Свои команды
\DeclareMathOperator{\sgn}{\mathop{sgn}}

%% Перенос знаков в формулах (по Львовскому)
\newcommand*{\hm}[1]{#1\nobreak\discretionary{}
{\hbox{$\mathsurround=0pt #1$}}{}}


% отступ для первого абзаца главы или параграфа
%\usepackage{indentfirst}

%%% Работа с картинками
\usepackage{graphicx}  % Для вставки рисунков
\graphicspath{{images/}{screnshots/}}  % папки с картинками
\DeclareGraphicsExtensions{.pdf,.png,.jpg}
\setlength\fboxsep{3pt} % Отступ рамки \fbox{} от рисунка
\setlength\fboxrule{1pt} % Толщина линий рамки \fbox{}
\usepackage{wrapfig} % Обтекание рисунков текстом

%%% Работа с таблицами
\usepackage{array,tabularx,tabulary,booktabs} % Дополнительная работа с таблицами
\usepackage{longtable}  % Длинные таблицы
\usepackage{multirow} % Слияние строк в таблице

%%% Теоремы
\theoremstyle{plain} % Это стиль по умолчанию, его можно не переопределять.
\newtheorem{theorem}{Теорема}[section]
\newtheorem{proposition}[theorem]{Утверждение}

\theoremstyle{plain} % Это стиль по умолчанию, его можно не переопределять.
\newtheorem{work}{Практическая работа}[part]


 
 
\theoremstyle{definition} % "Определение"
\newtheorem{corollary}{Следствие}[theorem]
\newtheorem{problem}{Задача}[section]
 
\theoremstyle{remark} % "Примечание"
\newtheorem*{nonum}{Решение}



%%% Программирование
\usepackage{etoolbox} % логические операторы

%%% Страница

%	\usepackage{fancyhdr} % Колонтитулы
% 	\pagestyle{fancy}
%   \renewcommand{\headrulewidth}{0pt}  % Толщина линейки, отчеркивающей верхний колонтитул
% 	\lfoot{Нижний левый}
% 	\rfoot{Нижний правый}
% 	\rhead{Верхний правый}
% 	\chead{Верхний в центре}
% 	\lhead{Верхний левый}
%	\cfoot{Нижний в центре} % По умолчанию здесь номер страницы

\usepackage{setspace} % Интерлиньяж
\onehalfspacing % Интерлиньяж 1.5
%\doublespacing % Интерлиньяж 2
%\singlespacing % Интерлиньяж 1

\usepackage{lastpage} % Узнать, сколько всего страниц в документе.

\usepackage{soul} % Модификаторы начертания


\usepackage[usenames,dvipsnames,svgnames,table,rgb]{xcolor}


\usepackage{csquotes} % Еще инструменты для ссылок

%\usepackage[style=authoryear,maxcitenames=2,backend=biber,sorting=nty]{biblatex}

\usepackage{multicol} % Несколько колонок

\usepackage{tikz} % Работа с графикой
\usepackage{pgfplots}
\usepackage{pgfplotstable}

% модуль для вставки рыбы
\usepackage{blindtext}

\usepackage{listings}
\usepackage{color}


% для поворота отдельной страницы. Использовать окружение \landscape
\usepackage{pdflscape} 
\usepackage{rotating} 


\definecolor{mygreen}{rgb}{0,0.6,0}
\definecolor{mygray}{rgb}{0.5,0.5,0.5}
\definecolor{mymauve}{rgb}{0.58,0,0.82}


% пример импорта файла
%\lstinputlisting{/home/denilai/repomy/conf/distributions}

\lstset{
	language=Python,
	basicstyle=\footnotesize,        % the size of the fonts that are used for the code
	numbers=left,                    % where to put the line-numbers; possible values are (none, left, right)
	numbersep=5pt,                   % how far the line-numbers are from the code
	numberstyle=\tiny\color{mygray}, % the style that is used for the line-numbers
	stepnumber=2,                    % the step between two line-numbers. If it's 1, each line will be numbered
	% Tab - 2 пробела
	tabsize=2,    
	% Автоматический перенос строк
	breaklines=true,
	frame=single,
	breakatwhitespace=true,
	title=\lstname 
}



\author{Кирилл Денисов}
\title{Лабораторная работа №1}
\date{\today}

% установка полуторного интервала
% \usepackage{setspace}  
% \onehalfspacing

% использовать Times New Roman
\renewcommand{\rmdefault}{ftm}
\renewcommand{\withouttheme}{1}

\begin{document}
	\thispagestyle{empty}
	% Вставка первого титульного листа
	%\newcommand{\withouttheme}{} добавить эту переменную для определения, нужна ли тема
%     {} - нужна
%    {1} - не нужна

%\newcommand{\withoutsubmissiondate}{} добавить эту переменную для определения, нужен ли срок предоставления отчета
%     {} - нужен
%    {1} - не нужен
\begin{center}
	\begin{figure}[h!]
		\begin{center}
		\includegraphics[width=0.17\linewidth]{\pathToCommonFolder/gerb}
		%\caption{}\label{pic:first}
		%	\vspace{5ex}
		\end{center}	
	\end{figure}
 	\small	МИНОБРНАУКИ РОССИИ \\
	Федеральное государственное бюджетное образовательное учреждение\\
						высшего профессионального образования\\
\normalsize					
\textbf{«МИРЭА – Российский технологический университет»\\
						РТУ МИРЭА}\\
						\noindent\rule{1\linewidth}{1pt}\\
       Институт информационных технологий\\ %\vspace{2ex}
					\kafedra\\
		\vspace{3ex}
			\large \textbf{\workname}  \\
		%\vspace{1ex}
						по дисциплине\\ «\discipline» \\
		\vspace{3ex}
		\if \withouttheme
			\textbf{Тема работы:}\\ <<\theme>>
		\fi
\vspace{3ex}
\small
\begin{table}[h!]
\begin{tabular}{lp{0.38\linewidth}p{0.2\linewidth}p{0.2\linewidth}}
	\textbf{Выполнил:} & студент группы ИВБО-02-19 & \studentfio &\includegraphics[width=\linewidth]{\pathToCommonFolder/signature}\\ \\
	\textbf{Принял:} & \rang & \teacherfio 
\end{tabular}
\end{table}
\end{center}

\begin{flushleft}
	\begin{tabular}{p{0.25\linewidth}l}

		Работа выполена & <<\noindent\rule{2em}{1pt}>>
		                    \noindent\rule{5em}{1pt} 202\noindent\rule{1em}{1pt} \\

		<<Зачтено>> & <<\noindent\rule{2em}{1pt}>>
		\noindent\rule{5em}{1pt} 202\noindent\rule{1em}{1pt} \\

	\end{tabular}
\end{flushleft}

\normalsize
\begin{center}	
\vfill 
Москва 2021
\end{center}

	\newpage
%	\tableofcontents
%	\newpage
	%\listoftables
	
\section*{Цель работы}
Исследовать работу с массивом и арифметические операции процессора
CPU580.
\section*{Индивидуальное задание. Вариант № 9}
\begin{problem*}
	Составить программу вычисления выражения:
	\begin{align}
		\sum_{i=1}^{n}a_i=a_1+a_2+\dots+a_n
	\end{align}

где $a_i$ --- число натурального ряда, начиная с 1;

n – количество чисел, при n = 90
\nonum
Описание используемых регистров (см. таблицу \ref{tab:registers}).

\begin{table}[h!]
	\centering
	\caption{Назначения регистров}
	\begin{tabular}{|m{0.1\linewidth}|p{0.5\linewidth}|}
		\hline
		Регистр & Назначение \\ \hline
		A &  Аккумулятор, где происходят все действия и результат\\ \hline
		B & Число для сравнения\\ \hline
		С & Прибавляемое число,
		выполняющее роль $X_i$ в
		исходной формуле\\\hline
	\end{tabular}
\label{tab:registers}
\end{table}


\begin{table}[h!]
	\centering
	\caption{Код программы}
	\begin{tabular}{|c|m{0.2\linewidth}|p{0.6\linewidth}|}
		\hline
		№ & \multicolumn{1}{c|}{Команда} & \multicolumn{1}{c|}{Описание} \\ \hline
		0 & \multicolumn{1}{c|}{JMP 1} & \multicolumn{1}{c|}{Прыжок по адресу 1} \\ \hline
		1 & \multicolumn{1}{c|}{MVI A,01} & Занесение значения 1 в регистр А (аккумулятор) \\ \hline
		\multicolumn{1}{|c|}{2} & \multicolumn{1}{c|}{MVI C,01} & Занесение значения 1 в регистр С \\ \hline
		3 &\multicolumn{1}{c|}{MVI B,5A} & Занесение значения 90 в регистр В \\ \hline
		\multicolumn{1}{|l|}{4} & \multicolumn{1}{c|}{
			INR C} & Инкремент регистра С \\ \hline
		5 & \multicolumn{1}{c|}{ADD C} & Сложение значения регистра С с аккумулятором(А) \\ \hline
		6 & \multicolumn{1}{c|}{CMP B} & Сравнение аккумулятора с В (через вычитание А из В) \\ \hline
		\multicolumn{1}{|l|}{7} & \multicolumn{1}{c|}{JNZ 4}  & {Если флаг нуля после сравнения не обнулился - прыгаем по адресу 4 в начало цикла} \\ \hline
		8 & \multicolumn{1}{c|}{HLT} & \multicolumn{1}{c|}{Задержка} \\ \hline
	\end{tabular}
	\label{tab:code}
\end{table}
\textbf{Описание алгоритма:}
Записываем в аккумулятор и в переменную, которую будем прибавлять, число 1 (1-2). В регистр для сравнения, по заданию, записываем число 90 (3) Увеличиваем $X_i$ (4) и прибавляем к аккумулятору (5). Если Xi = 90, конец
программы, иначе перейти на шаг 4. Подробное описание шагов приведено в таблице \ref{tab:code}
\newpage
\end{problem*}

\begin{problem*}
	Составить программу деления однобайтных двоичных чисел.
	
	\nonum
\end{problem*}
Описание используемых регистров приведены в таблице \ref{tab:registers2}.
\begin{table}[h!]
	\centering
	\caption{Назначение регистров}
	\begin{tabular}{|c|c|}
		\hline
		Регистр & Назначение \\ \hline
		B & Делитель \\ \hline
		C & Делимое \\ \hline
		D & Остаток \\ \hline
		E & \multicolumn{1}{l|}{Счетчик цикла} \\ \hline
		H & Результат \\ \hline
	\end{tabular}
	\label{tab:registers2}
\end{table}

\begin{table}[h!]
	\caption{Код программы}
	\centering
	\begin{tabular}{|r|r|p{0.5\linewidth}|}
		\hline
		0 & MVI E 07 & Счетчик цикла \\ \hline
		1 & LXI B, N1, N2 & Загружаем из памяти делимое - число N1, делитель - число N2 \\ \hline
		2 & MOV A, C & \\ \cline{1-2}
		3 & RAL & Сдвиг делимого   \\ \cline{1-2}
		4 & MOV C, A &  \\ \hline
		5 & MOV A, D & Сдвигаем значение частичного остатка \\ \hline
		6 & RAL &  \\ \hline
		7 & SUB B & Вычитаем делитель \\ \hline
		8 & JNC 10 & Если происходит переполнение - восстанавливаем значение частичного остатка \\\cline{1-2}
		9 & ADD B &  \\ \hline
		10 & MOV D, A & Возвращаем ЧО в регистр \\ \hline
		11 & CMC & Инвертируем перенос, так как если он произошел, то произошло переполнение, а значит вычитание делителя из ЧО нельзя производить \\ \hline
		12 & MOV A, H &  \\ \cline{1-2}
		13 & RAL & Запоминаем перенос \\ \cline{1-2}
		14 & MOV H, A &  \\ \hline
		15 & DCR E & Уменьшаем счетчик циклов \\ \hline
		16 & JNZ 2 & Цикл - пока счетчик не равен 0 \\ \hline
		17 & HLT & Иначе, конец программы \\ \hline
	\end{tabular}
	\label{tab:code2}
\end{table}

\textbf{Описание алгоритма:}
В данном алгоритме деление происходит практически также как при делении в столбик. Мы берем число с разрядностью вдвое больше исходного делимого и начинаем вычитать делитель, начиная со старшего разряда и каждый раз сдвигаясь к младшим. При вычитании, если мы получаем отрицательный результат, значит частичный остаток все ещё больше делителя, поэтому необходимо восстановить его до прежнего значения и продолжить выполнение (шаг 8-9). Подробное описание шагов приведено в таблице \ref{tab:code2}.

\newpage
\paragraph{Вывод}
В ходе данной практической работы мы научились реализовывать простые алгоритмы при помощи языка ассемблера CPU580. Алгоритм деления не является оптимальным, но при этом является более наглядным и простым для понимания, что важно, учитывая ознакомительных характер работы.


\end{document}


