\documentclass[a4paper,14pt]{extarticle}

% Путь до папки с общими шаблонами
\newcommand{\pathToCommonFolder}{/home/denilai/Documents/repos/latex/Common}
% Название работы в титуле
\newcommand{\workname}{Отчет по практической работе №7}
% Название дисциплины в титуле
\newcommand{\discipline}{Моделирование бизнес-процессов}
% Название кафедры в титуле
\newcommand{\kafedra}{Кафедра практической и прикладной информатики (ППИ)}
% Тема работы в титуле
\newcommand{\theme}{}
\newcommand{\dontneedtheme}{1}
% Должность преподавателя в титуле
\newcommand{\rang}{}
% ФИО студента в титуле
\newcommand{\studentfio}{К.~Ю.~Денисов}
% ФИО преподавателя в титуле
\newcommand{\teacherfio}{Х.~Г.~Ахмедова}


\usepackage{tabularx}


\usepackage{booktabs}
\newcolumntype{b}{X}
\newcolumntype{s}{>{\hsize=.5\hsize}X}
\newcommand{\heading}[1]{\multicolumn{1}{c}{#1}}

% установка размера шрифта для всего документа
%\fontsize{20pt}{18pt}\selectfont
\usepackage{extsizes} % Возможность сделать 14-й шрифт

% Вставка заготовки преамбулы
% Этот шаблон документа разработан в 2014 году
% Данилом Фёдоровых (danil@fedorovykh.ru) 
% для использования в курсе 
% <<Документы и презентации в \LaTeX>>, записанном НИУ ВШЭ
% для Coursera.org: http://coursera.org/course/latex .
% Исходная версия шаблона --- 
% https://www.writelatex.com/coursera/latex/5.3

% В этом документе преамбула

% Для корректного использования русских символов в формулах
% пакеты hyperref и настройки, связанные с ним, стоит загуржать
% перед загрузкой пакета mathtext



% поддержка русских букв
% кодировка шрифта
%\usepackage[T2A]{fontenc} 
\usepackage{pscyr}

% использование ненумеровонного абзаца с добавлением его в содержаниеl

\newcommand{\anonsection}[1]{\section*{#1}\addcontentsline{toc}{section}{#1}}
\newcommand{\sectionunderl}[1]{\section*{\underline{#1}}}


% настройка окружения enumerate
\usepackage{enumitem}
\setlist{noitemsep}
\setlist[enumerate]{labelsep=*, leftmargin=1.5pc}

\usepackage{hyperref}

% сначала ставить \usepackage{extsizes} % Возможность сделать 14-й шрифт
% для корректной установки полей вставлять преамбулу следует в последнюю очередь (но перед дерективой замены \rmdefault)
\usepackage[top=20mm,bottom=25mm,left=35mm,right=15mm]{geometry} % Простой способ задавать поля

\hypersetup{				% Гиперссылки
	unicode=true,           % русские буквы в раздела PDF
	pdftitle={Заголовок},   % Заголовок
	pdfauthor={Автор},      % Автор
	pdfsubject={Тема},      % Тема
	pdfcreator={Создатель}, % Создатель
	pdfproducer={Производитель}, % Производитель
	pdfkeywords={keyword1} {key2} {key3}, % Ключевые слова
	colorlinks=true,       	% false: ссылки в рамках; true: цветные ссылки
	linkcolor=red,          % внутренние ссылки
	citecolor=black,        % на библиографию
	filecolor=magenta,      % на файлы
	urlcolor=blue           % на URL
}

%%% Работа с русским языком
\usepackage{cmap}					% поиск в PDF
\usepackage{mathtext} 				% русские буквы в формулах
\usepackage[T2A]{fontenc}			% кодировка
\usepackage[utf8]{inputenc}			% кодировка исходного текста
\usepackage[english,russian]{babel}	% локализация и переносы
\usepackage{indentfirst}
\frenchspacing

%для изменения названия списка иллюстраций
\usepackage{tocloft}


\renewcommand{\epsilon}{\ensuremath{\varepsilon}}
\renewcommand{\phi}{\ensuremath{\varphi}}
\renewcommand{\kappa}{\ensuremath{\varkappa}}
\renewcommand{\le}{\ensuremath{\leqslant}}
\renewcommand{\leq}{\ensuremath{\leqslant}}
\renewcommand{\ge}{\ensuremath{\geqslant}}
\renewcommand{\geq}{\ensuremath{\geqslant}}
\renewcommand{\emptyset}{\varnothing}

% Изменения параметров списка иллюстраций
\renewcommand{\cftfigfont}{Рисунок } % добавляем везде "Рисунок" перед номером
\addto\captionsrussian{\renewcommand\listfigurename{Список иллюстративного материала}}

\newcommand{\tm}{\texttrademark\ }
\newcommand{\reg}{\textregistered\ }




%требования к спискам, возможно пригодится
%---------------------------------------------------------
%Требования на списки в стандарте следующие:
%нумерованные списки на первом уровне помечаются как «а)», «б)», «в)»… На втором — как «1)», «2)», «3)». Да-да, я тоже не вижу тут ни капли логики.
%ненумерованные списки помечаются дефисами.

%\usepackage{enumitem}
%\makeatletter
%\AddEnumerateCounter{\asbuk}{\@asbuk}{м)}
%\makeatother
%\setlist{nolistsep}
%\renewcommand{\labelitemi}{-}
%\renewcommand{\labelenumi}{\asbuk{enumi})}
%\renewcommand{\labelenumii}{\arabic{enumii})}
%------------------------------------------------------


%%% Дополнительная работа с математикой
\usepackage{amsmath,amsfonts,amssymb,amsthm,mathtools} % AMS
\usepackage{icomma} % "Умная" запятая: $0,2$ --- число, $0, 2$ --- перечисление

%% Номера формул
%\mathtoolsset{showonlyrefs=true} % Показывать номера только у тех формул, на которые есть \eqref{} в тексте.
%\usepackage{leqno} % Нумереация формул слева

%% Свои команды
\DeclareMathOperator{\sgn}{\mathop{sgn}}

%% Перенос знаков в формулах (по Львовскому)
\newcommand*{\hm}[1]{#1\nobreak\discretionary{}
{\hbox{$\mathsurround=0pt #1$}}{}}


% отступ для первого абзаца главы или параграфа
%\usepackage{indentfirst}

%%% Работа с картинками
\usepackage{graphicx}  % Для вставки рисунков
\graphicspath{{images/}{screnshots/}}  % папки с картинками
\DeclareGraphicsExtensions{.pdf,.png,.jpg}
\setlength\fboxsep{3pt} % Отступ рамки \fbox{} от рисунка
\setlength\fboxrule{1pt} % Толщина линий рамки \fbox{}
\usepackage{wrapfig} % Обтекание рисунков текстом

%%% Работа с таблицами
\usepackage{array,tabularx,tabulary,booktabs} % Дополнительная работа с таблицами
\usepackage{longtable}  % Длинные таблицы
\usepackage{multirow} % Слияние строк в таблице

%%% Теоремы
\theoremstyle{plain} % Это стиль по умолчанию, его можно не переопределять.
\newtheorem{theorem}{Теорема}[section]
\newtheorem{proposition}[theorem]{Утверждение}

\theoremstyle{plain} % Это стиль по умолчанию, его можно не переопределять.
\newtheorem{work}{Практическая работа}[part]


 
 
\theoremstyle{definition} % "Определение"
\newtheorem{corollary}{Следствие}[theorem]
\newtheorem{problem}{Задача}[section]
 
\theoremstyle{remark} % "Примечание"
\newtheorem*{nonum}{Решение}



%%% Программирование
\usepackage{etoolbox} % логические операторы

%%% Страница

%	\usepackage{fancyhdr} % Колонтитулы
% 	\pagestyle{fancy}
%   \renewcommand{\headrulewidth}{0pt}  % Толщина линейки, отчеркивающей верхний колонтитул
% 	\lfoot{Нижний левый}
% 	\rfoot{Нижний правый}
% 	\rhead{Верхний правый}
% 	\chead{Верхний в центре}
% 	\lhead{Верхний левый}
%	\cfoot{Нижний в центре} % По умолчанию здесь номер страницы

\usepackage{setspace} % Интерлиньяж
\onehalfspacing % Интерлиньяж 1.5
%\doublespacing % Интерлиньяж 2
%\singlespacing % Интерлиньяж 1

\usepackage{lastpage} % Узнать, сколько всего страниц в документе.

\usepackage{soul} % Модификаторы начертания


\usepackage[usenames,dvipsnames,svgnames,table,rgb]{xcolor}


\usepackage{csquotes} % Еще инструменты для ссылок

%\usepackage[style=authoryear,maxcitenames=2,backend=biber,sorting=nty]{biblatex}

\usepackage{multicol} % Несколько колонок

\usepackage{tikz} % Работа с графикой
\usepackage{pgfplots}
\usepackage{pgfplotstable}

% модуль для вставки рыбы
\usepackage{blindtext}

\usepackage{listings}
\usepackage{color}


% для поворота отдельной страницы. Использовать окружение \landscape
\usepackage{pdflscape} 
\usepackage{rotating} 


\definecolor{mygreen}{rgb}{0,0.6,0}
\definecolor{mygray}{rgb}{0.5,0.5,0.5}
\definecolor{mymauve}{rgb}{0.58,0,0.82}


% пример импорта файла
%\lstinputlisting{/home/denilai/repomy/conf/distributions}

\lstset{
	language=Python,
	basicstyle=\footnotesize,        % the size of the fonts that are used for the code
	numbers=left,                    % where to put the line-numbers; possible values are (none, left, right)
	numbersep=5pt,                   % how far the line-numbers are from the code
	numberstyle=\tiny\color{mygray}, % the style that is used for the line-numbers
	stepnumber=2,                    % the step between two line-numbers. If it's 1, each line will be numbered
	% Tab - 2 пробела
	tabsize=2,    
	% Автоматический перенос строк
	breaklines=true,
	frame=single,
	breakatwhitespace=true,
	title=\lstname 
}



\author{Кирилл Денисов}
\title{Практическая работа №4}
\date{\today}

\renewcommand{\withouttheme}{1}
\renewcommand{\withoutsubmissiondate}{}

% установка полуторного интервала
% \usepackage{setspace}  
% \onehalfspacing

% использовать Times New Roman
\renewcommand{\rmdefault}{ftm}


\begin{document}
	\thispagestyle{empty}
	% Вставка первого титульного листа
	%\newcounter{withouttheme}

%\setcounter{withouttheme}{<n>} установить значение счетчика  withouttheme для определения, нужна ли тема
%    {0} - нужна
%    {1} - не нужна

%\setcounter{withoutsubmissiondate}{<n>} установить значение счетчика  withoutsubmissiondate для определения, нужна ли дата представления к защите
%     {0} - нужна
%     {1} - не нужена
\begin{center}
	\begin{figure}[h!]
		\begin{center}
		%\vspace{-10ex}
		\includegraphics[width=0.17\linewidth]{\pathToCommonFolder/gerb}
		%\caption{}\label{pic:first}
		%	\vspace{5ex}
		\end{center}	
	\end{figure}
 	\small	МИНОБРНАУКИ РОССИИ \\
	Федеральное государственное бюджетное образовательное учреждение\\
						высшего образования\\
\normalsize					
\textbf{«МИРЭА – Российский технологический университет»\\
						РТУ МИРЭА}\\
						\noindent\rule{1\linewidth}{1pt}\\
       Институт информационных технологий\\ %\vspace{2ex}
					\kafedra\\
		\vspace{3ex}
			\large \textbf{\workname}  \\
		%\vspace{1ex}
						по дисциплине\\ «\discipline» \\
		\vspace{3ex}
		\ifnum \value{withouttheme}=0 {
			\textbf{Тема работы:}\\ <<\theme>>
		}
		\else {}
		\fi
\vspace{10ex}
\small
\begin{table}[h!]
\begin{tabular}{lp{0.6\linewidth}l}
	\textbf{Выполнил:} & студент группы ИВБО-02-19 & \\ 
	& & \studentfio \\%Д.~Н.~Федосеев\\%А.~М.~Сосунов\\%К.~Ю.~Денисов\\%И.~А.~Кремнев
	\textbf{Принял:} & \rang & \\
	& & \teacherfio \hfill\\
\end{tabular}
\end{table}
\end{center}
\ifnum \value{withoutsubmissiondate}=0 {
	\begin{flushleft}
		Работа представлена к защите <<\rule{3ex}{1pt}>>\rule{10ex}{1pt} 202\rule{1ex}{1pt} г.\hfill
	\end{flushleft}
\else {}
\fi

\normalsize
\begin{center}	
\vfill
Москва 2022
\end{center}

	\newpage
%	\tableofcontents
	\newpage
	%\listoftables
\anonsection{Ход работы}

\subsection*{Цель работы}
Задание предназначено для самостоятельного выполнения студентами с
целью закрепления навыка правильной формулировки названия выходов каждого
этапа и работы выполняемого бизнес-процесса.

\subsection*{Постановка задачи}
Сформировать описания процессов.

\subsection*{Индивидуальный вариант 6}

\subsubsection*{Результат работы}
\problem{Построить схему бизнес-процесса на уровне предприятия и на уровне процессов
	«Получение чистой прибыли по разработке ПО»:}
\task{Построить схему бизнес-процесса на уровне предприятия и на уровне процессов «Заключить договор»}

\begin{table}[htbp]
	\begin{center}
		\begin{tabular}{|l|l|l|}
			\hline
			\textbf{Название процесса} & Заключить договор &  \\ \hline
			\textbf{Владелец} & Менеджер по работе с клиентами &  \\ \hline
			\textbf{Вход} & Заявка на заключение договора &  \\ \hline
			\textbf{Выход} & Заключенный договор (с  условиями) &  \\ \hline
			\multicolumn{ 1}{|l|}{\textbf{Ресурсы}} & Человеческий ресурс &  \\ \cline{ 2- 3}
			\multicolumn{ 1}{|l|}{} & Материальные ресурсы &  \\ \cline{ 2- 3}
			\multicolumn{ 1}{|l|}{} & Информационные ресурсы &  \\ \hline
			\textbf{Поставщик} & Заказчик & Внешний \\ \hline
			\textbf{Получатель} & Менеджер & Внутренний \\ \hline
		\end{tabular}
	\end{center}
	\label{}
\end{table}

\if

\begin{table}[h!]
	\small
	\begin{tabular}{|p{0.22\linewidth}|p{0.54\linewidth}|p{0.15\linewidth}|}
		\hline
		\centerboldcell{Название процесса} & {Получение чистой прибыли от разработки ПО} &  \\ \hline
		\textbf{Владелец} & Бухгалтер &  \\ \hline
		\textbf{Вход} & Договор с указанием денежных средств &  \\ \hline
		\textbf{Выход} & Сумма чистой прибыли &  \\ \hline
		\textbf{Ресурсы} & Человеческие и материальные ресурсы &  \\ \hline
		\textbf{Поставщик} & Заказчик & Внешний \\ \hline
		\textbf{Получатель} & Генеральный директор & Внутренний \\ \hline
	\end{tabular}
	\label{}
\end{table}
\normalsize
\fi
\newpage
\task{Построить схему бизнес-процесса на уровне предприятия и на уровне процессов
	«Разработка ПО»}

\begin{table}[h!]
	\begin{center}
		\begin{tabular}{|l|l|l|}
			\hline
			\textbf{Название процесса} & Разработка ПО &  \\ \hline
			\textbf{Владелец} & Менеджер проекта &  \\ \hline
			\textbf{Вход} & Договор &  \\ \hline
			\textbf{Выход} & Разработанное ПО &  \\ \hline
			\multicolumn{ 1}{|l|}{\textbf{Ресурсы}} & Человеческий ресурс &  \\ \cline{ 2- 3}
			\multicolumn{ 1}{|l|}{} & Материальные ресурсы &  \\ \cline{ 2- 3}
			\multicolumn{ 1}{|l|}{} & Информационные ресурсы &  \\ \hline
			\textbf{Поставщик} & Отдел по работе с клиентами & Внутренний \\ \hline
			\textbf{Получатель} & Заказчик & Внешний \\ \hline
		\end{tabular}
	\end{center}
	\label{}
\end{table}

\if

\begin{table}[htbp]
	\begin{center}
		\begin{tabular}{|l|l|l|}
			\hline
			\textbf{Название процесса} & Заключить договор &  \\ \hline
			\textbf{Владелец} & Менеджер по работе с клиентами &  \\ \hline
			\textbf{Вход} & Заявка на договор &  \\ \hline
			\textbf{Выход} & Заключенный договор &  \\ \hline
			\textbf{Ресурсы} & Человеческие и материальные ресурсы & \textbf{} \\ \hline
			\textbf{Поставщик} & Создатель ПО & Внешний \\ \hline
			\textbf{Получатель} & Менеджер проекта & Внутренний \\ \hline
		\end{tabular}
	\end{center}
	\label{}
\end{table}
\newpage
\fi

\newpage
\problem{Сформировать таблицу подпроцесса «Внедрение ПО», определить внешних и
	внутренних поставщиков и пользователей:}

\begin{table}[h!]
	\begin{center}
		\begin{tabular}{|l|p{0.4\linewidth}|l|}
			\hline
			\textbf{Название процесса} & Внедрение ПО &  \\ \hline
			\textbf{Владелец} & {Менеджер   отдела по работе с клиентами} &  \\ \hline
			\multicolumn{ 1}{|l|}{\textbf{Вход}} & Договор &  \\ \cline{ 2- 3}
			\multicolumn{ 1}{|l|}{} & Необученный персонал &  \\ \cline{ 2- 3}
			\multicolumn{ 1}{|l|}{} & Разработанное ПО &  \\ \hline
			\multicolumn{ 1}{|l|}{\textbf{Выход}} & Обученный персонал &  \\ \cline{ 2- 3}
			\multicolumn{ 1}{|l|}{} & Внедренное ПО &  \\ \hline
			\multicolumn{ 1}{|l|}{\textbf{Ресурсы}} & Человеческий ресурс &  \\ \cline{ 2- 3}
			\multicolumn{ 1}{|l|}{} & Материальные ресурсы &  \\ \cline{ 2- 3}
			\multicolumn{ 1}{|l|}{} & Информационные ресурсы &  \\ \hline
			\multicolumn{ 1}{|l|}{\textbf{Поставщик}} & Менеджер проекта & Внутренний \\ \hline
			\textbf{Получатель} & Заказчик & Внешний \\ \hline
		\end{tabular}
	\end{center}
	\label{}
\end{table}


\if
\newpage

\begin{table}[htbp]
	\begin{center}
		\begin{tabular}{|l|l|l|}
			\hline
			\textbf{Название процесса} & Разработка ПО &  \\ \hline
			\textbf{Владелец} & Менеджерпроекта &  \\ \hline
			\textbf{Вход} & Договор &  \\ \hline
			\textbf{Выход} & Разработанное ПО &  \\ \hline
			\textbf{Ресурсы} & Человеческие и материальные ресурсы &  \\ \hline
			\textbf{Поставщик} & Менеджер по работе с клиентами & Внутренний \\ \hline
			\textbf{Получатель} & Заказчик & Внешний \\ \hline
		\end{tabular}
	\end{center}
	\label{}
\end{table}
\fi

\problem{Описание процесса по каскадной модели по разработке ПО:}

\vspace{3ex}
\begin{longtable}{|l|l|}

	\endfirsthead
	
	\hline
	\multicolumn{2}{|c|}{\textit{Продолжение таблицы}}\\
	\hline
	\endhead
	
	\hline
	\endfoot
	
	
	\hline
	\multicolumn{2}{| c |}{\textit{Конец таблицы}}\\
	\hline\hline
	\endlastfoot

	\hline
	\textbf{Название процесса} & Планирование \\ \hline
	\textbf{Цель процесса} & Формирование плана работы \\ \hline
	\textbf{Владелец процесса} & Менеджер проекта \\ \hline
	\multicolumn{ 1}{|c|}{\textbf{Участники процесса}} & Технолог разработки ПО \\ \cline{ 2- 2}
	\multicolumn{ 1}{|l|}{} & Заказчик \\ \hline
	\textbf{Вход} & Заключенный договор \\ \hline
	\textbf{Выход} & План \\ \hline
	\textbf{Название процесса} & Формирование требований \\ \hline
	\textbf{Цель процесса} & Формирование спецификации \\ \hline
	\textbf{Владелец процесса} & Менеджер проекта \\ \hline
	\multicolumn{ 1}{|c|}{\textbf{Участники процесса}} & Технолог разработки ПО \\ \cline{ 2- 2}
	\multicolumn{ 1}{|l|}{} & Бизнес-аналитик \\ \cline{ 2- 2}
	\multicolumn{ 1}{|l|}{} & Заказчик ПО \\ \cline{ 2- 2}
	\multicolumn{ 1}{|l|}{} & Будущие пользователи \\ \hline
	\textbf{Вход} & План \\ \hline
	\textbf{Выход} & Спецификация \\ \hline
	\textbf{Название процесса} & Анализ и проектирование \\ \hline
	\textbf{Цель процесса} & Создание дизайна \\ \hline
	\textbf{Владелец процесса} & Менеджер проекта \\ \hline
	\multicolumn{ 1}{|c|}{\textbf{Участники процесса}} & Проектировщик \\ \cline{ 2- 2}
	\multicolumn{ 1}{|l|}{} & бизнес-аналитик \\ \cline{ 2- 2}
	\multicolumn{ 1}{|l|}{} & разработчик \\ \cline{ 2- 2}
	\multicolumn{ 1}{|l|}{} & будущие пользователи \\ \cline{ 2- 2}
	\multicolumn{ 1}{|l|}{} & заказчик ПО \\ \hline
	\textbf{Вход} & Спецификация \\ \hline
	\textbf{Выход} & Дизайн \\ \hline
	\textbf{Название процесса} & Конструирование \\ \hline
	\textbf{Цель процесса} & Разработка кода \\ \hline
	\textbf{Владелец процесса} & Менеджер проекта \\ \hline
	\textbf{Участники процесса} & разработчик, \\ \hline
	\textbf{} & инженер по качеству, \\ \hline
	\textbf{} & технолог разработки ПО \\ \hline
	\textbf{Вход} & Дизайн \\ \hline
	\textbf{Выход} & Код \\ \hline
	\textbf{Название процесса} & Интеграция и тестирование \\ \hline
	\textbf{Цель процесса} & Интеграция и тестирование продукта \\ \hline
	\textbf{Владелец процесса} & Менеджер проекта \\ \hline
	\multicolumn{ 1}{|c|}{\textbf{Участники процесса}} & Разработчик \\ \cline{ 2- 2}
	\multicolumn{ 1}{|l|}{} & инженер по качеству \\ \cline{ 2- 2}
	\multicolumn{ 1}{|l|}{} & тестировщик и техническим писателем \\ \hline
	\textbf{Вход} & Код \\ \hline
	\textbf{Выход} & Продукт \\ \hline
	\textbf{Название процесса} & Поддержка и эксплуатация \\ \hline
	\textbf{Цель процесса} & Поддержка и эксплуатация готового продукта \\ \hline
	\textbf{Владелец процесса} & Менеджер проекта \\ \hline
	\multicolumn{ 1}{|c|}{\textbf{Участники процесса}} & заказчик ПО \\ \cline{ 2- 2}
	\multicolumn{ 1}{|l|}{} & менеджер по работе с клиентами \\ \cline{ 2- 2}
	\multicolumn{ 1}{|l|}{} & сотрудник сервисного отдела \\ \hline
	\textbf{Вход} & Продукт \\ \hline
	\textbf{Выход} & Готовое ПО 
	\label{long}
\end{longtable}

\problem{Формирование выходов по каскадной модели:}

\begin{table}[h!]
	\begin{center}
		\begin{tabular}{|l|p{0.47\linewidth}|}
			\hline
			\textbf{Процесс} & Выход \\ \hline
			\textbf{Исследование концепции} & Исследованная концепция \\ \hline
			\textbf{Выработка требований} & Выработанные требования \\ \hline
			\textbf{Проектирование} & Создание проекта \\ \hline
			\textbf{Реализация компонент} & Код \\ \hline
			\textbf{Интеграция компонент} & Готовая программа \\ \hline
		\end{tabular}
	\end{center}
	\label{}
\end{table}

\newpage

\problem{На основе основных этапов и работ, выполняемых при разработке экономической информационной системы, определить выходы каждого основного этапа бизнес-процесса «Разработка ЭИС»:}

\begin{table}[h!]
	\begin{center}
		\begin{tabular}{|l|l|}
			\hline
			\textbf{Процесс} & Выход \\ \hline
			\textbf{Техническое задание} & Разработанное техническое задание \\ \hline
			\textbf{Эскизный проект} & Созданный эскиз \\ \hline
			\textbf{Технический проект} & Разработанный технический процесс \\ \hline
			\textbf{Рабочий проект} & Созданный  рабочий проект \\ \hline
			\textbf{Тестирование программы} & Протестированная программа \\ \hline
			\textbf{Внедрение} & Готовая программная документация по ЭИС \\ \hline
		\end{tabular}
	\end{center}
	\label{}
\end{table}



\problem{На основе основных этапов и работ, выполняемых при разработке ЭИС, определить выходы каждой стадии бизнес-процесса «Разработка ЭИС»:}

\begin{table}[h!]
	\begin{center}
		\begin{tabular}{|l|p{0.47\linewidth}|}
			\hline
			\textbf{Процесс} & Выход \\ \hline
			\textbf{Предпроектная стадия разработки} & Разработанное техническое задание \\ \hline
			\textbf{Проектирование ЭИС} &{Описание  технологии  работы  с макетом ЭИС} \\ \hline
			\textbf{Разработка ЭИС} & Готовый ЭИС \\ \hline
			\textbf{Внедрение} & Внедренные  ЭИС  в  компании- заказчика \\ \hline
		\end{tabular}
	\end{center}
	\label{}
\end{table}



\newpage

% TODO: \usepackage{graphicx} required

\anonsection{Список использованных\\источников и литературы}
\begin{enumerate}
	\item https://github.com/Vitaliy-Yakovchuk/ramus
	\item Информационные технологии. Анализ и проектирование информационных систем: учебное пособие / Рочев К.В. --- Санкт-Петербург: Лань, 2019. --- 128

	
	\item Войнов И.В., Пудовкина С.Г., Телегин А.И. Моделирование экономических систем и процессов. Опыт построения ARIS-моделей: Монография. --- Челябинск: Изд. ЮУрГУ, 2002. --- 392 с.
	
	
	\item Григорьев Д. Моделирование бизнес-процессов предприятия. Режим доступа: http://www.valex.net/articles/process.html. --- Загл. с экрана.
	
	\item Калянов Г.Н. Моделирование, анализ, реорганизация и автоматизация бизнес-процессов // М.: Финансы и статистика, 2006.
\end{enumerate}


%\newpage
%{\centering
%	\anonsection{ПРИЛОЖЕНИЕ А}
%}
%\label{A}

\end{document}



