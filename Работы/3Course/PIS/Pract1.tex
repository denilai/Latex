\documentclass[a4paper,14pt]{extarticle}

\newcommand{\stend}{\textbf{Wb-demo-kit v.2}}

% Путь до папки с общими шаблонами
\newcommand{\pathToCommonFolder}{/home/denilai/Documents/repos/latex/Common}

% Название работы в титуле
\newcommand{\workname}{Отчет по практическим работам}
% Название дисциплины в титуле
\newcommand{\discipline}{Проектирование информационных систем}
% Название кафедры в титуле
\newcommand{\kafedra}{Кафедра инструментального и прикладного программного обеспечения}
% Тема работы в титуле
\newcommand{\theme}{Формирование требований к системе}
% Должность преподавателя в титуле
\newcommand{\rang}{ассистент}

% ФИО студента в титуле
\newcommand{\studentfio}{К.~Ю.~Денисов}%\\Д.~Н.~Федосеев\\А.~М.~Сосунов}\\%К.~Ю.~Денисов\\%И.~А.~Кремнев
% ФИО преподавателя в титуле
\newcommand{\teacherfio}{А.~А.~Русляков}


\usepackage{tabularx}
\usepackage{lastpage}


\usepackage{booktabs}
\newcolumntype{b}{X}
\newcolumntype{s}{>{\hsize=.5\hsize}X}
\newcommand{\heading}[1]{\multicolumn{1}{c}{#1}}

% установка размера шрифта для всего документа
%\fontsize{20pt}{18pt}\selectfont
\usepackage{extsizes} % Возможность сделать 14-й шрифт

% Вставка заготовки преамбулы
% Этот шаблон документа разработан в 2014 году
% Данилом Фёдоровых (danil@fedorovykh.ru) 
% для использования в курсе 
% <<Документы и презентации в \LaTeX>>, записанном НИУ ВШЭ
% для Coursera.org: http://coursera.org/course/latex .
% Исходная версия шаблона --- 
% https://www.writelatex.com/coursera/latex/5.3

% В этом документе преамбула

% Для корректного использования русских символов в формулах
% пакеты hyperref и настройки, связанные с ним, стоит загуржать
% перед загрузкой пакета mathtext



% поддержка русских букв
% кодировка шрифта
%\usepackage[T2A]{fontenc} 
\usepackage{pscyr}

% использование ненумеровонного абзаца с добавлением его в содержаниеl

\newcommand{\anonsection}[1]{\section*{#1}\addcontentsline{toc}{section}{#1}}
\newcommand{\sectionunderl}[1]{\section*{\underline{#1}}}


% настройка окружения enumerate
\usepackage{enumitem}
\setlist{noitemsep}
\setlist[enumerate]{labelsep=*, leftmargin=1.5pc}

\usepackage{hyperref}

% сначала ставить \usepackage{extsizes} % Возможность сделать 14-й шрифт
% для корректной установки полей вставлять преамбулу следует в последнюю очередь (но перед дерективой замены \rmdefault)
\usepackage[top=20mm,bottom=25mm,left=35mm,right=15mm]{geometry} % Простой способ задавать поля

\hypersetup{				% Гиперссылки
	unicode=true,           % русские буквы в раздела PDF
	pdftitle={Заголовок},   % Заголовок
	pdfauthor={Автор},      % Автор
	pdfsubject={Тема},      % Тема
	pdfcreator={Создатель}, % Создатель
	pdfproducer={Производитель}, % Производитель
	pdfkeywords={keyword1} {key2} {key3}, % Ключевые слова
	colorlinks=true,       	% false: ссылки в рамках; true: цветные ссылки
	linkcolor=red,          % внутренние ссылки
	citecolor=black,        % на библиографию
	filecolor=magenta,      % на файлы
	urlcolor=blue           % на URL
}

%%% Работа с русским языком
\usepackage{cmap}					% поиск в PDF
\usepackage{mathtext} 				% русские буквы в формулах
\usepackage[T2A]{fontenc}			% кодировка
\usepackage[utf8]{inputenc}			% кодировка исходного текста
\usepackage[english,russian]{babel}	% локализация и переносы
\usepackage{indentfirst}
\frenchspacing

%для изменения названия списка иллюстраций
\usepackage{tocloft}


\renewcommand{\epsilon}{\ensuremath{\varepsilon}}
\renewcommand{\phi}{\ensuremath{\varphi}}
\renewcommand{\kappa}{\ensuremath{\varkappa}}
\renewcommand{\le}{\ensuremath{\leqslant}}
\renewcommand{\leq}{\ensuremath{\leqslant}}
\renewcommand{\ge}{\ensuremath{\geqslant}}
\renewcommand{\geq}{\ensuremath{\geqslant}}
\renewcommand{\emptyset}{\varnothing}

% Изменения параметров списка иллюстраций
\renewcommand{\cftfigfont}{Рисунок } % добавляем везде "Рисунок" перед номером
\addto\captionsrussian{\renewcommand\listfigurename{Список иллюстративного материала}}

\newcommand{\tm}{\texttrademark\ }
\newcommand{\reg}{\textregistered\ }




%требования к спискам, возможно пригодится
%---------------------------------------------------------
%Требования на списки в стандарте следующие:
%нумерованные списки на первом уровне помечаются как «а)», «б)», «в)»… На втором — как «1)», «2)», «3)». Да-да, я тоже не вижу тут ни капли логики.
%ненумерованные списки помечаются дефисами.

%\usepackage{enumitem}
%\makeatletter
%\AddEnumerateCounter{\asbuk}{\@asbuk}{м)}
%\makeatother
%\setlist{nolistsep}
%\renewcommand{\labelitemi}{-}
%\renewcommand{\labelenumi}{\asbuk{enumi})}
%\renewcommand{\labelenumii}{\arabic{enumii})}
%------------------------------------------------------


%%% Дополнительная работа с математикой
\usepackage{amsmath,amsfonts,amssymb,amsthm,mathtools} % AMS
\usepackage{icomma} % "Умная" запятая: $0,2$ --- число, $0, 2$ --- перечисление

%% Номера формул
%\mathtoolsset{showonlyrefs=true} % Показывать номера только у тех формул, на которые есть \eqref{} в тексте.
%\usepackage{leqno} % Нумереация формул слева

%% Свои команды
\DeclareMathOperator{\sgn}{\mathop{sgn}}

%% Перенос знаков в формулах (по Львовскому)
\newcommand*{\hm}[1]{#1\nobreak\discretionary{}
{\hbox{$\mathsurround=0pt #1$}}{}}


% отступ для первого абзаца главы или параграфа
%\usepackage{indentfirst}

%%% Работа с картинками
\usepackage{graphicx}  % Для вставки рисунков
\graphicspath{{images/}{screnshots/}}  % папки с картинками
\DeclareGraphicsExtensions{.pdf,.png,.jpg}
\setlength\fboxsep{3pt} % Отступ рамки \fbox{} от рисунка
\setlength\fboxrule{1pt} % Толщина линий рамки \fbox{}
\usepackage{wrapfig} % Обтекание рисунков текстом

%%% Работа с таблицами
\usepackage{array,tabularx,tabulary,booktabs} % Дополнительная работа с таблицами
\usepackage{longtable}  % Длинные таблицы
\usepackage{multirow} % Слияние строк в таблице

%%% Теоремы
\theoremstyle{plain} % Это стиль по умолчанию, его можно не переопределять.
\newtheorem{theorem}{Теорема}[section]
\newtheorem{proposition}[theorem]{Утверждение}

\theoremstyle{plain} % Это стиль по умолчанию, его можно не переопределять.
\newtheorem{work}{Практическая работа}[part]


 
 
\theoremstyle{definition} % "Определение"
\newtheorem{corollary}{Следствие}[theorem]
\newtheorem{problem}{Задача}[section]
 
\theoremstyle{remark} % "Примечание"
\newtheorem*{nonum}{Решение}



%%% Программирование
\usepackage{etoolbox} % логические операторы

%%% Страница

%	\usepackage{fancyhdr} % Колонтитулы
% 	\pagestyle{fancy}
%   \renewcommand{\headrulewidth}{0pt}  % Толщина линейки, отчеркивающей верхний колонтитул
% 	\lfoot{Нижний левый}
% 	\rfoot{Нижний правый}
% 	\rhead{Верхний правый}
% 	\chead{Верхний в центре}
% 	\lhead{Верхний левый}
%	\cfoot{Нижний в центре} % По умолчанию здесь номер страницы

\usepackage{setspace} % Интерлиньяж
\onehalfspacing % Интерлиньяж 1.5
%\doublespacing % Интерлиньяж 2
%\singlespacing % Интерлиньяж 1

\usepackage{lastpage} % Узнать, сколько всего страниц в документе.

\usepackage{soul} % Модификаторы начертания


\usepackage[usenames,dvipsnames,svgnames,table,rgb]{xcolor}


\usepackage{csquotes} % Еще инструменты для ссылок

%\usepackage[style=authoryear,maxcitenames=2,backend=biber,sorting=nty]{biblatex}

\usepackage{multicol} % Несколько колонок

\usepackage{tikz} % Работа с графикой
\usepackage{pgfplots}
\usepackage{pgfplotstable}

% модуль для вставки рыбы
\usepackage{blindtext}

\usepackage{listings}
\usepackage{color}


% для поворота отдельной страницы. Использовать окружение \landscape
\usepackage{pdflscape} 
\usepackage{rotating} 


\definecolor{mygreen}{rgb}{0,0.6,0}
\definecolor{mygray}{rgb}{0.5,0.5,0.5}
\definecolor{mymauve}{rgb}{0.58,0,0.82}


% пример импорта файла
%\lstinputlisting{/home/denilai/repomy/conf/distributions}

\lstset{
	language=Python,
	basicstyle=\footnotesize,        % the size of the fonts that are used for the code
	numbers=left,                    % where to put the line-numbers; possible values are (none, left, right)
	numbersep=5pt,                   % how far the line-numbers are from the code
	numberstyle=\tiny\color{mygray}, % the style that is used for the line-numbers
	stepnumber=2,                    % the step between two line-numbers. If it's 1, each line will be numbered
	% Tab - 2 пробела
	tabsize=2,    
	% Автоматический перенос строк
	breaklines=true,
	frame=single,
	breakatwhitespace=true,
	title=\lstname 
}



\author{Кирилл Денисов}
\title{Лабораторная работа №1}
\date{\today}

\setcounter{withouttheme}{0}
\setcounter{withoutsubmissiondate}{1}

%если нужна тема работы в отчете, то указать в скобках что-либо, иначе оаставить пустым
%\renewcommand{\withouttheme}{}
%если нужна дата представления отчета, то указать в скобках что-либо
%\renewcommand{\withoutsubmissiondate}{1}

% установка полуторного интервала
% \usepackage{setspace}  
% \onehalfspacing

% использовать Times New Roman
\renewcommand{\rmdefault}{ftm}


\newcommand{\tb}{ThingsBoard~}

\begin{document}
	\thispagestyle{empty}
	% Вставка первого титульного листа
	% Есть две версии титульного листа - одиночный (titul) и групповой (titulAll)
	%\newcounter{withouttheme}

%\setcounter{withouttheme}{<n>} установить значение счетчика  withouttheme для определения, нужна ли тема
%    {0} - нужна
%    {1} - не нужна

%\setcounter{withoutsubmissiondate}{<n>} установить значение счетчика  withoutsubmissiondate для определения, нужна ли дата представления к защите
%     {0} - нужна
%     {1} - не нужена
\begin{center}
	\begin{figure}[h!]
		\begin{center}
		%\vspace{-10ex}
		\includegraphics[width=0.17\linewidth]{\pathToCommonFolder/gerb}
		%\caption{}\label{pic:first}
		%	\vspace{5ex}
		\end{center}	
	\end{figure}
 	\small	МИНОБРНАУКИ РОССИИ \\
	Федеральное государственное бюджетное образовательное учреждение\\
						высшего образования\\
\normalsize					
\textbf{«МИРЭА – Российский технологический университет»\\
						РТУ МИРЭА}\\
						\noindent\rule{1\linewidth}{1pt}\\
       Институт информационных технологий\\ %\vspace{2ex}
					\kafedra\\
		\vspace{3ex}
			\large \textbf{\workname}  \\
		%\vspace{1ex}
						по дисциплине\\ «\discipline» \\
		\vspace{3ex}
		\ifnum \value{withouttheme}=0 {
			\textbf{Тема работы:}\\ <<\theme>>
		}
		\else {}
		\fi
\vspace{10ex}
\small
\begin{table}[h!]
\begin{tabular}{lp{0.6\linewidth}l}
	\textbf{Выполнил:} & студент группы ИВБО-02-19 & \\ 
	& & \studentfio \\%Д.~Н.~Федосеев\\%А.~М.~Сосунов\\%К.~Ю.~Денисов\\%И.~А.~Кремнев
	\textbf{Принял:} & \rang & \\
	& & \teacherfio \hfill\\
\end{tabular}
\end{table}
\end{center}
\ifnum \value{withoutsubmissiondate}=0 {
	\begin{flushleft}
		Работа представлена к защите <<\rule{3ex}{1pt}>>\rule{10ex}{1pt} 202\rule{1ex}{1pt} г.\hfill
	\end{flushleft}
\else {}
\fi

\normalsize
\begin{center}	
\vfill
Москва 2022
\end{center}

	\newpage
	\tableofcontents
	\newpage
	%\listoftables
	
\normalsize

\section{Практическая работа №1: <<Формирование требований к системе>>}
\subsection{РЕФЕРАТ}
Суть данной практической работы заключается в анализе и формировании требований к разрабатываемой информационной системе <<Электронный сборник лабораторных работ>>, в том числе требований к составу и содержанию работ по подготовке объекта автоматизации к вводу системы в действие, а также требований к документированию определении и формализации бизнес-ролей.

Данная практическая работа содержит
\pageref*{LastPage}~страницы
%, \totfig~рисунков
, \tottab~таблицы.

\newpage
\subsection{ОПИСАНИЕ ОБЪЕКТА АВТОМАТИЗАЦИИ}

Разрабатываемая информационная система <<Электронный сборник лабораторных работ>> (далее по тексту Система) служит для сбора, хранения и учета письменных работ различного характера, выполненных учащимися средних общеобразовательных, средних специальных и высших учебных заведений. На данной платформе пользователям на безвозмездной основе предоставляется доступ к загруженным документам, топикам и темам форума и медиафайлам. 

Система выполняет функции образовательной платформы. Платформа может быть интегрирована в информационную среду учебных заведений, предоставляя инструменты для учета и хранения работ учащихся. 

\newpage
\subsection{ОБЩИЕ СВЕДЕНИЯ}
\subsubsection{Список требований и определений}

Подсистема управления доступом (ПУД) --- часть Системы, назначающая разрешения конечным пользователям в зависимости от их роли в вашей организации. ПУД обеспечивает гранулярный контроль, предлагая простой и управляемый подход к управлению доступом, который менее подвержен ошибкам, чем индивидуальное назначение разрешений.

\subsubsection{Описание бизнес-ролей}
\begin{table}[h!]
	\caption{Описание бизнес-ролей}\label{tab:roles}
	\begin{tabular}{|c|p{0.6\linewidth}|}
		\hline 
		\textbf{Бизнес-роль} & \multicolumn{1}{c|}{\textbf{Описание}}\\ \hline
		Администратор системы & Лицо, имеющее доступ к ПУД и другим инструментам администрирования Системы \\ \hline
		Пользователь & Лицо или организация, которая использует Систему для размещения и хранения файлов, имеет личный кабинет на портале, может участвовать в обсуждениях на форуме\\ \hline
	\end{tabular}
\end{table}


\newpage
\subsection{ТРЕБОВАНИЯ К СИСТЕМЕ}
\subsubsection{Бизнес требования}

Система должна предоставлять функционал для удобного удаленного хранения и управления электронными документами.

Использование системы в качестве хранилища должно быть более привлекательно, чем хранение файлов на локальном компьютере или физическом носителе.

Система должна иметь возможность быть интегрированной в информационную платформу общеобразовательных учреждений.


\subsubsection{Пользовательские требования}
Пользователь должен иметь возможность авторизоваться в системе для загрузки документов в хранилище.

Пользователь должен иметь возможность скачивания файлов без ограничений и необходимости регистрации .

Пользователь должен иметь возможность оставлять отзыв и оценку конкретному документу, размещенному в системе.

Пользователь должен иметь возможность создать тему на портале.

\subsubsection{Функциональные требования}

Система должна отвечать требованиям по хранению, обработке и защите персональных данных в соответствии с 152-ФЗ «О персональных данных».

Система должна иметь упорядоченную структуру файлов с разделением по языкам, научным дисциплинам, видам работ.

Данные должны хранится в течение 20 лет.

В системе должен сочетаться функционал хранилища данных и образовательного форума.

Платформа должна реализовывать систему управления (авторизации) доступом на основе ролей.

Система должна реализовывать функционал по аутентификации пользователя по логину (почтовому адресу) и паролю.

Персональные данные пользователя и хеш-строки паролей должны храниться в зашифрованном виде.

К загрузке должны допускаться файлы любого формата.

\subsubsection{Нефункциональные требования}

В качестве алгоритма хеширования должен использоваться SHA-256.

Система должна представлять собой интернет-ресурс, разработанный в среде разработки Bitrix Framework.

Каждому пользователю должно предоставляться персональное хранилище данных в размере 5 ГБ с возможностью расширения. 

Для хранения сведений о пользователях системы, записях образовательного форума и метаданных используется база данных под управлением PostgreSQL.

Для хранения пользовательских файлов должно использоваться хранилище блочного типа. Файлы в таком хранилище делятся на блоки равного размера и размещаются в памяти сервера. По запросу платформы система хранения собирает файл из блоков, используя метаданные.

\subsubsection{Требования к пользовательскому интерфейсу}
Пользовательский интерфейс должен содержать нейтральные цвета и
контрастный, хорошо читаемый текст. Основные цвета – белый, синий.

Обязательными элементами пользовательская  являются:

\textbf{Наличие функции поиска}.

 Является обязательным условием каждого крупного веб-проекта, состоящего более чем из 10-и страниц.
	
	Реализация простого поиска, позволяющий находить нужную информацию по ключевым запросам. Поисковое окно располагается в верхней части сайта и проходит через все страницы сайта.
	
\textbf{Меню}
	
	Список доступных разделов и категорий должен быть всегда заметен пользователю, независимо от его местонахождения на сайте. Следовательно, пользователь всегда должен видеть возможные варианты переходов.
	Кроме того, подобное «сквозное» меню упрощает индексацию сайта для поисковых машин, способствует равномерному распределению «веса» между всеми страницами сайта.
	
	Веб-интерфейс должен содержать элементы управления и манипуляции документами, вкладки и разделы, отведенные под форум.
	
	 \textbf{Адаптивный дизайн}
	
	Дизайн веб-страниц должен обеспечивать правильное отображение сайта на различных устройствах, подключённых к интернету и динамически подстраиваться под заданные размеры окна браузера.
	
	Сайт может работать на смартфоне, планшете, ноутбуке и телевизоре с выходом в интернет, то есть практически на всем спектре устройств.
	
\subsubsection{Требования к защите информации}
Для обеспечения защиты информации от несанкционированного доступа требуется реализация следующих функций:

\paragraph*{В части управления доступом:}

\begin{itemize}
	\item  должна осуществляться идентификация и проверка подлинности субъектов доступа при входе в Систему по идентификатору (коду) и паролю условно-постоянного действия длиной не менее шести символов;
	
	\item должна осуществляться идентификация АРМ, серверов, каналов связи, внешних устройств ЭВМ по логическим именам;
	
	\item должен осуществляться контроль доступа субъектов к защищаемым ресурсам в соответствии с ролевой моделью.
\end{itemize}
\paragraph*{В части регуляции и учета:}
\begin{itemize}
	\item  должна осуществляться регистрация входа/выхода субъектов доступа в систему/из системы.
	\item В параметрах регистрации должны указываться:
	\subitem время и дата входа/выхода субъекта доступа в систему/из системы
	\subitem идентификатор (код или фамилия) субъекта, предъявленный при попытке доступа;
\end{itemize}

\subsubsection{Требования по сохранности информации при авариях}
Сохранность информации в Системе должна обеспечиваться при следующих аварийных ситуациях:
\begin{itemize}
	\item импульсные помехи, сбои и перерывы в электропитании;
	\item нарушение или выход из строя каналов связи локальной сети;
	\item сбой общего или специального программного обеспечения (сервера);
	\item ошибки в работе персонала
\end{itemize}


\subsubsection{Требования по стандартизации и унификации}
Для работы с БД должен использоваться язык SQL в рамках стандарта ANSI SQL-92.

Для разработки пользовательского интерфейса и средств генерации отчетов (любых твердых копий) должны использоваться языки 4-го поколения.

При создании Системы должно использоваться стандартное общее программное обеспечение, включающее лицензионные ОС, СУБД, сетевую операционную систему.



\subsubsection{Требования к безопасности}
Система не должна выдавать варианты, которые потенциально могут
усугубить состояние оборудования.

Система должна выдавать пользователю правила технической безопасности, которые ему необходимо принять, чтобы пользоваться системой.

Системой должен осуществляться контроль доступа к конфиденциальным данным.

Должны предприниматься шаги, направленные на предотвращение утечек персональных данных пользователей Системы.

Система должна быть защищена от внешних атак.

\subsubsection{Требования к документированию}
Проектная документация должна быть разработана в соответствии с
ГОСТ 34.201-89 и ГОСТ 7.32-2017.

Отчетные материалы должны включать в себя текстовые материалы
(представленные в виде бумажной копии и на цифровом носителе) и графические материалы.

Предоставить следующие документы:

\begin{enumerate}
	\item Схема функциональной структуры автоматизируемой деятельности.
	\item  Описание технологического процесса обработки данных.
	\item  Описание информационного обеспечения.
	\item  Описание программного обеспечения АС.
	\item  Схема логической структуры БД.
	\item  Руководство пользователя.
	\item  Описание контрольного примера (по ГОСТ 24.102).
	\item  Протокол испытаний (по ГОСТ 24.102).
	
\end{enumerate}


\subsubsection{Требования к функциям, выполняемым системой}
\begin{table}[h!]
	\caption{\label{tab:functions} Требования к функциям, выполняемым системой.}
	\begin{center}\small
		\begin{tabular}{|m{0.4\linewidth}|m{0.4\linewidth}|}
			\hline
			\textbf{Функция} & \textbf{Задача}\\
			\hline
			\multirow{3}{0.95\linewidth}{Добавление новых документов в хранилище данных} 
			& При необходимости добавить пользователя в Систему  \\\cline{2-2}
			& Авторизовать пользователя \\\cline{2-2}
			& Заполнить сведения о документе \\\cline{2-2}
			& Загрузить документ в объектное хранилище данных \\\cline{2-2}
			& Обновить сведения о документах пользователя \\\cline{2-2}
			& Провести проверку и корректировку связей в хранилище метаданных \\\hline
			
			\multirow{4}{0.95\linewidth}{Загрузка документов из хранилища данных} 
			& Запросить сведения о документе \\\cline{2-2}
			& Выгрузить документ из объектного хранилища данных \\\cline{2-2}
			& Выдать результат пользователю \\\cline{2-2}
			& Провести проверку и корректировку связей в хранилище метаданных \\\hline
		\end{tabular}
	\end{center}
\end{table}
\newpage
\subsection{СЦЕНАРИИ ИСПОЛЬЗОВАНИЯ}


\begin{enumerate}
	\item Пользователь заходит на веб ресурс с целью ознакомления с содержанием или скачивания интересующего его документа. Он может оценить размещенный документ, оставив отметку «Нравится». Также пользователь может получить цифровую копию документа не проходя аутентификацию в системе. 
	\item Пользователь заходит на веб-ресурс с целью загрузки документа. Для этого ему необходимо пройти регистрацию в Системе. После этого пользователь получит доступ к личному кабинету и персональному хранилищу данных размером 5 ГБ.
	\item Пользователь заходит на веб-ресурс с целью заведения темы на форуме, входящем в состав Системы. Для этого ему необходимо пройти регистрацию в Системе. После этого пользователь получит доступ к личному кабинету и получит возможность оставлять комментарии под записями на форуме, создавать новые темы для обсуждения.
\end{enumerate}

\end{document}

