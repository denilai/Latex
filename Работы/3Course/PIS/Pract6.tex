\documentclass[a4paper,14pt]{extarticle}

\newcommand{\stend}{\textbf{Wb-demo-kit v.2}}

% Путь до папки с общими шаблонами
\newcommand{\pathToCommonFolder}{/home/denilai/Documents/repos/latex/Common}

% Название работы в титуле
\newcommand{\workname}{Отчет по практической работе №6}
% Название дисциплины в титуле
\newcommand{\discipline}{Проектирование информационных систем}
% Название кафедры в титуле
\newcommand{\kafedra}{Кафедра инструментального и прикладного программного обеспечения}
% Тема работы в титуле
\newcommand{\theme}{Проектирование структуры данных информационной системы и создание ER-диаграммы}
% Должность преподавателя в титуле
\newcommand{\rang}{ассистент}

% ФИО студента в титуле
\newcommand{\studentfio}{К.~Ю.~Денисов}%\\Д.~Н.~Федосеев\\А.~М.~Сосунов}\\%К.~Ю.~Денисов\\%И.~А.~Кремнев
% ФИО преподавателя в титуле
\newcommand{\teacherfio}{А.~А.~Русляков}


\usepackage{tabularx}
\usepackage{lastpage}


\usepackage{booktabs}
\newcolumntype{b}{X}
\newcolumntype{s}{>{\hsize=.5\hsize}X}
\newcommand{\heading}[1]{\multicolumn{1}{|c|}{\textbf{#1}}}

% установка размера шрифта для всего документа
%\fontsize{20pt}{18pt}\selectfont
\usepackage{extsizes} % Возможность сделать 14-й шрифт

% Вставка заготовки преамбулы
% Этот шаблон документа разработан в 2014 году
% Данилом Фёдоровых (danil@fedorovykh.ru) 
% для использования в курсе 
% <<Документы и презентации в \LaTeX>>, записанном НИУ ВШЭ
% для Coursera.org: http://coursera.org/course/latex .
% Исходная версия шаблона --- 
% https://www.writelatex.com/coursera/latex/5.3

% В этом документе преамбула

% Для корректного использования русских символов в формулах
% пакеты hyperref и настройки, связанные с ним, стоит загуржать
% перед загрузкой пакета mathtext



% поддержка русских букв
% кодировка шрифта
%\usepackage[T2A]{fontenc} 
\usepackage{pscyr}

% использование ненумеровонного абзаца с добавлением его в содержаниеl

\newcommand{\anonsection}[1]{\section*{#1}\addcontentsline{toc}{section}{#1}}
\newcommand{\sectionunderl}[1]{\section*{\underline{#1}}}


% настройка окружения enumerate
\usepackage{enumitem}
\setlist{noitemsep}
\setlist[enumerate]{labelsep=*, leftmargin=1.5pc}

\usepackage{hyperref}

% сначала ставить \usepackage{extsizes} % Возможность сделать 14-й шрифт
% для корректной установки полей вставлять преамбулу следует в последнюю очередь (но перед дерективой замены \rmdefault)
\usepackage[top=20mm,bottom=25mm,left=35mm,right=15mm]{geometry} % Простой способ задавать поля

\hypersetup{				% Гиперссылки
	unicode=true,           % русские буквы в раздела PDF
	pdftitle={Заголовок},   % Заголовок
	pdfauthor={Автор},      % Автор
	pdfsubject={Тема},      % Тема
	pdfcreator={Создатель}, % Создатель
	pdfproducer={Производитель}, % Производитель
	pdfkeywords={keyword1} {key2} {key3}, % Ключевые слова
	colorlinks=true,       	% false: ссылки в рамках; true: цветные ссылки
	linkcolor=red,          % внутренние ссылки
	citecolor=black,        % на библиографию
	filecolor=magenta,      % на файлы
	urlcolor=blue           % на URL
}

%%% Работа с русским языком
\usepackage{cmap}					% поиск в PDF
\usepackage{mathtext} 				% русские буквы в формулах
\usepackage[T2A]{fontenc}			% кодировка
\usepackage[utf8]{inputenc}			% кодировка исходного текста
\usepackage[english,russian]{babel}	% локализация и переносы
\usepackage{indentfirst}
\frenchspacing

%для изменения названия списка иллюстраций
\usepackage{tocloft}


\renewcommand{\epsilon}{\ensuremath{\varepsilon}}
\renewcommand{\phi}{\ensuremath{\varphi}}
\renewcommand{\kappa}{\ensuremath{\varkappa}}
\renewcommand{\le}{\ensuremath{\leqslant}}
\renewcommand{\leq}{\ensuremath{\leqslant}}
\renewcommand{\ge}{\ensuremath{\geqslant}}
\renewcommand{\geq}{\ensuremath{\geqslant}}
\renewcommand{\emptyset}{\varnothing}

% Изменения параметров списка иллюстраций
\renewcommand{\cftfigfont}{Рисунок } % добавляем везде "Рисунок" перед номером
\addto\captionsrussian{\renewcommand\listfigurename{Список иллюстративного материала}}

\newcommand{\tm}{\texttrademark\ }
\newcommand{\reg}{\textregistered\ }




%требования к спискам, возможно пригодится
%---------------------------------------------------------
%Требования на списки в стандарте следующие:
%нумерованные списки на первом уровне помечаются как «а)», «б)», «в)»… На втором — как «1)», «2)», «3)». Да-да, я тоже не вижу тут ни капли логики.
%ненумерованные списки помечаются дефисами.

%\usepackage{enumitem}
%\makeatletter
%\AddEnumerateCounter{\asbuk}{\@asbuk}{м)}
%\makeatother
%\setlist{nolistsep}
%\renewcommand{\labelitemi}{-}
%\renewcommand{\labelenumi}{\asbuk{enumi})}
%\renewcommand{\labelenumii}{\arabic{enumii})}
%------------------------------------------------------


%%% Дополнительная работа с математикой
\usepackage{amsmath,amsfonts,amssymb,amsthm,mathtools} % AMS
\usepackage{icomma} % "Умная" запятая: $0,2$ --- число, $0, 2$ --- перечисление

%% Номера формул
%\mathtoolsset{showonlyrefs=true} % Показывать номера только у тех формул, на которые есть \eqref{} в тексте.
%\usepackage{leqno} % Нумереация формул слева

%% Свои команды
\DeclareMathOperator{\sgn}{\mathop{sgn}}

%% Перенос знаков в формулах (по Львовскому)
\newcommand*{\hm}[1]{#1\nobreak\discretionary{}
{\hbox{$\mathsurround=0pt #1$}}{}}


% отступ для первого абзаца главы или параграфа
%\usepackage{indentfirst}

%%% Работа с картинками
\usepackage{graphicx}  % Для вставки рисунков
\graphicspath{{images/}{screnshots/}}  % папки с картинками
\DeclareGraphicsExtensions{.pdf,.png,.jpg}
\setlength\fboxsep{3pt} % Отступ рамки \fbox{} от рисунка
\setlength\fboxrule{1pt} % Толщина линий рамки \fbox{}
\usepackage{wrapfig} % Обтекание рисунков текстом

%%% Работа с таблицами
\usepackage{array,tabularx,tabulary,booktabs} % Дополнительная работа с таблицами
\usepackage{longtable}  % Длинные таблицы
\usepackage{multirow} % Слияние строк в таблице

%%% Теоремы
\theoremstyle{plain} % Это стиль по умолчанию, его можно не переопределять.
\newtheorem{theorem}{Теорема}[section]
\newtheorem{proposition}[theorem]{Утверждение}

\theoremstyle{plain} % Это стиль по умолчанию, его можно не переопределять.
\newtheorem{work}{Практическая работа}[part]


 
 
\theoremstyle{definition} % "Определение"
\newtheorem{corollary}{Следствие}[theorem]
\newtheorem{problem}{Задача}[section]
 
\theoremstyle{remark} % "Примечание"
\newtheorem*{nonum}{Решение}



%%% Программирование
\usepackage{etoolbox} % логические операторы

%%% Страница

%	\usepackage{fancyhdr} % Колонтитулы
% 	\pagestyle{fancy}
%   \renewcommand{\headrulewidth}{0pt}  % Толщина линейки, отчеркивающей верхний колонтитул
% 	\lfoot{Нижний левый}
% 	\rfoot{Нижний правый}
% 	\rhead{Верхний правый}
% 	\chead{Верхний в центре}
% 	\lhead{Верхний левый}
%	\cfoot{Нижний в центре} % По умолчанию здесь номер страницы

\usepackage{setspace} % Интерлиньяж
\onehalfspacing % Интерлиньяж 1.5
%\doublespacing % Интерлиньяж 2
%\singlespacing % Интерлиньяж 1

\usepackage{lastpage} % Узнать, сколько всего страниц в документе.

\usepackage{soul} % Модификаторы начертания


\usepackage[usenames,dvipsnames,svgnames,table,rgb]{xcolor}


\usepackage{csquotes} % Еще инструменты для ссылок

%\usepackage[style=authoryear,maxcitenames=2,backend=biber,sorting=nty]{biblatex}

\usepackage{multicol} % Несколько колонок

\usepackage{tikz} % Работа с графикой
\usepackage{pgfplots}
\usepackage{pgfplotstable}

% модуль для вставки рыбы
\usepackage{blindtext}

\usepackage{listings}
\usepackage{color}


% для поворота отдельной страницы. Использовать окружение \landscape
\usepackage{pdflscape} 
\usepackage{rotating} 


\definecolor{mygreen}{rgb}{0,0.6,0}
\definecolor{mygray}{rgb}{0.5,0.5,0.5}
\definecolor{mymauve}{rgb}{0.58,0,0.82}


% пример импорта файла
%\lstinputlisting{/home/denilai/repomy/conf/distributions}

\lstset{
	language=Python,
	basicstyle=\footnotesize,        % the size of the fonts that are used for the code
	numbers=left,                    % where to put the line-numbers; possible values are (none, left, right)
	numbersep=5pt,                   % how far the line-numbers are from the code
	numberstyle=\tiny\color{mygray}, % the style that is used for the line-numbers
	stepnumber=2,                    % the step between two line-numbers. If it's 1, each line will be numbered
	% Tab - 2 пробела
	tabsize=2,    
	% Автоматический перенос строк
	breaklines=true,
	frame=single,
	breakatwhitespace=true,
	title=\lstname 
}



\author{Кирилл Денисов}
\title{Лабораторная работа №1}
\date{\today}

\setcounter{withouttheme}{0}
\setcounter{withoutsubmissiondate}{1}

%если нужна тема работы в отчете, то указать в скобках что-либо, иначе оаставить пустым
%\renewcommand{\withouttheme}{}
%если нужна дата представления отчета, то указать в скобках что-либо
%\renewcommand{\withoutsubmissiondate}{1}

% установка полуторного интервала
% \usepackage{setspace}  
% \onehalfspacing

% использовать Times New Roman
\renewcommand{\rmdefault}{ftm}


\newcommand{\tb}{ThingsBoard~}

\begin{document}
	\thispagestyle{empty}
	% Вставка первого титульного листа
	% Есть две версии титульного листа - одиночный (titul) и групповой (titulAll)
	%\newcounter{withouttheme}

%\setcounter{withouttheme}{<n>} установить значение счетчика  withouttheme для определения, нужна ли тема
%    {0} - нужна
%    {1} - не нужна

%\setcounter{withoutsubmissiondate}{<n>} установить значение счетчика  withoutsubmissiondate для определения, нужна ли дата представления к защите
%     {0} - нужна
%     {1} - не нужена
\begin{center}
	\begin{figure}[h!]
		\begin{center}
		%\vspace{-10ex}
		\includegraphics[width=0.17\linewidth]{\pathToCommonFolder/gerb}
		%\caption{}\label{pic:first}
		%	\vspace{5ex}
		\end{center}	
	\end{figure}
 	\small	МИНОБРНАУКИ РОССИИ \\
	Федеральное государственное бюджетное образовательное учреждение\\
						высшего образования\\
\normalsize					
\textbf{«МИРЭА – Российский технологический университет»\\
						РТУ МИРЭА}\\
						\noindent\rule{1\linewidth}{1pt}\\
       Институт информационных технологий\\ %\vspace{2ex}
					\kafedra\\
		\vspace{3ex}
			\large \textbf{\workname}  \\
		%\vspace{1ex}
						по дисциплине\\ «\discipline» \\
		\vspace{3ex}
		\ifnum \value{withouttheme}=0 {
			\textbf{Тема работы:}\\ <<\theme>>
		}
		\else {}
		\fi
\vspace{10ex}
\small
\begin{table}[h!]
\begin{tabular}{lp{0.6\linewidth}l}
	\textbf{Выполнил:} & студент группы ИВБО-02-19 & \\ 
	& & \studentfio \\%Д.~Н.~Федосеев\\%А.~М.~Сосунов\\%К.~Ю.~Денисов\\%И.~А.~Кремнев
	\textbf{Принял:} & \rang & \\
	& & \teacherfio \hfill\\
\end{tabular}
\end{table}
\end{center}
\ifnum \value{withoutsubmissiondate}=0 {
	\begin{flushleft}
		Работа представлена к защите <<\rule{3ex}{1pt}>>\rule{10ex}{1pt} 202\rule{1ex}{1pt} г.\hfill
	\end{flushleft}
\else {}
\fi

\normalsize
\begin{center}	
\vfill
Москва 2022
\end{center}

	\newpage
	%\tableofcontents
	\newpage
	%\listoftables
	
\normalsize

\section{Разработка диаграммы отношений (Entity Relation Diagram)}
\subsection{Цель работы}
Логическое моделирование ИС <<Электронный сборник лабораторных работ>>.
\subsection{Моделирование системы}
Для логического моделирования было выбрано программное обеспечение Draw.io в силу того, что оно предлагает удобный веб-интерфейс для работы а также реализует возможность создания диаграмм в различных нотациях. Данное программное обеспечение позволяет экспортировать файлы в формате, удобном для переноса между платформами и в виде изображений, что удобно при разработке документации, прилагаемой к реализуемой ИС.

\subsection{Краткая постановка задачи}
Главная задача системы --- сбор и обработка научных и ученических работ пользователей. Система должна представлять данные о файлах в структурированном виде, предлагать удобный интерфейс взаимодействия с личными файлами и представлять возможность получения файлов, хранящихся в системе.

Основываясь на поставленной задаче, была создана модель данных базы метаданных, обеспечивающей работу и обслуживание объектного хранилища, содержащего пользовательские файлы, загружаемые в систему.

Хранилище метаданных выполнено по схеме <<Звезда>>, где основная таблица фактов связана с несколькими таблицами измерений, организуя удобную для хранения многомерных показателей схему реалиционных таблиц.
\newpage
В таблице фактов содержатся следующие данные:

\begin{table}[h!]
	\caption{Таблица фактов PaperFacts}
	\begin{tabular}{|p{0.3\linewidth}|p{0.6\linewidth}|}
		\hline
		\heading{Название} & \heading{Назначение} \\ \hline
		ID & Индефикатор записи \\ \hline
		paperID & Индефикатор документа \\ \hline
		authorID & Индефикатор автора \\ \hline
		disciplineID & Индефикатор дисциплины  \\ \hline
		work\_typeID & Индефикатор типа работы \\ \hline
		organizationID & Индефикатор организации \\ \hline
		licenseID & Индефикатор лицензии \\ \hline
		timestamp & Временная метка \\ \hline
		status & Статус для версионирования \\ \hline
		name & Название работы \\ \hline
	\end{tabular}
	\label{tab:facts}
\end{table}

В таблице измерения <<Авторы>> содержатся следующие данные:

\begin{table}[h!]
	\caption{Таблица измерения Authors}
	\begin{tabular}{|p{0.3\linewidth}|p{0.6\linewidth}|}
		\hline
		\heading{Название} & \heading{Назначение} \\ \hline
		authorID & Индефикатор автора \\ \hline
		firstname & Имя \\ \hline
		secondname & Фамилия \\ \hline
		degree & Степень (должность) \\ \hline
		nickname & Никнейм \\ \hline
		email & Адрес электронной почты \\ \hline
		country & Страна \\ \hline
		city & Город \\ \hline
	\end{tabular}
	\label{tab:authors}
\end{table}

В таблице измерения <<Организации>> содержатся следующие данные:
\begin{table}[h!]
	\caption{Таблица измерения Organizations}
	\begin{tabular}{|p{0.3\linewidth}|p{0.6\linewidth}|}
		\hline
		\heading{Название} & \heading{Назначение} \\ \hline
		organizationID & Индефикатор организации \\ \hline
		country & Страна \\ \hline
		city & Город \\ \hline
		type & Тип организации \\ \hline
		name & Название организации \\ \hline
	\end{tabular}
	\label{tab:orgs}
\end{table}

\newpage
В таблице измерения <<Лицензии>> содержатся следующие данные:
\begin{table}[h!]
	\caption{Таблица измерения Licenses}
	\begin{tabular}{|p{0.3\linewidth}|p{0.6\linewidth}|}
		\hline
		\heading{Название} & \heading{Назначение} \\ \hline
		licenseID & Индефикатор лицензии \\ \hline
		name & Название лицензии \\ \hline
		supervisor & Котролирующая организация \\ \hline
	\end{tabular}
	\label{tab:licenses}
\end{table}

В таблице измерения <<Типы работ>> содержатся следующие данные:
\begin{table}[h!]
\caption{Таблица измерения WorkTypes}
\begin{tabular}{|p{0.3\linewidth}|p{0.6\linewidth}|}
	\hline
	\heading{Название} & \heading{Назначение} \\ \hline
		work\_typeID & Индефикатор типа работы \\ \hline
		name & Название типа работы \\ \hline
	\end{tabular}
	\label{tab:worktypes}
\end{table}


В таблице измерения <<Работы>> содержатся следующие данные:
\begin{table}[h!]
	\caption{Таблица измерения Papers}
	\begin{tabular}{|p{0.3\linewidth}|p{0.6\linewidth}|}
		\hline
		\heading{Название} & \heading{Назначение} \\ \hline
		paperID & Индефикатор файла работы \\ \hline
		URI & Универсальный индефикатор ресурса \\ \hline
		status & Статус для версионирования \\ \hline
	\end{tabular}
	\label{tab:papers}
\end{table}

В таблице измерения <<Дисциплины>> содержатся следующие данные:
\begin{table}[h!]
	\caption{Таблица измерения Disciplines}
	\begin{tabular}{|p{0.3\linewidth}|p{0.6\linewidth}|}
		\hline
		\heading{Название} & \heading{Назначение} \\ \hline
		disciplineID & Индефикатор дисциплины  \\ \hline
		name & Название дисциплины \\ \hline
	\end{tabular}
	\label{tab:disciplines}
\end{table}

Таблица фактов поддерживает версионирование медленно изменяющихся измерений второго типа (SCD2). Версионирование нужно не для хранения разных версий одного файла в системе, а для отражения в базе данных факта изменения параметров файла. Например, при изменении типа работы запись об этом сохранится в базе, при этом это никак не скажется на доступности файла в объектном хранилище, так как ссылка на файловый ресурс не будет затронута.

ER диаграмма приведена на рисунке \ref{fig:data-model}.

% TODO: \usepackage{graphicx} required
\begin{figure}[h!]
	\centering
	\includegraphics[width=0.8\linewidth]{images/data-model}
	\caption{ER диаграмма}
	\label{fig:data-model}
\end{figure}





Разработанный пример ER-диаграммы является примером
концептуальной диаграммы, не учитывающей особенности конкретной
СУБД. 

На основе данной концептуальной диаграммы можно построить
физическую диаграмму, которая будут учитывать такие особенности СУБД,
как допустимые типы, наименования полей и таблиц, ограничения
целостности и т.п.
Для преобразования концептуальной модели в физическую необходимо
знать, что:
\begin{enumerate}
	\item Каждая сущность в ER-диаграмме представляет собой таблицу
	базы данных;
	\item Каждый атрибут становится колонкой (полем) соответствующей
	таблицы;
	\item  В некоторых таблицах необходимо вставить новые атрибуты
	(поля), которых не было в концептуальной модели — это ключевые атрибуты 
	родительских таблиц, перемещённых в дочерние таблицы для того, чтобы
	обеспечить связь между таблицами посредством внешних ключей.
\end{enumerate}

\section*{Вывод}

В ходе данной практической работы была спроектирована и описана модель данных Подсистемы Хранения информационной системы <<Сборник лабораторных работ>>. Определена схема необходимых сущностей (таблиц) и взаимосвязь таблиц между собой. По ходу разработки Системы данная схема будет дополняться и уточняться.
\end{document}

