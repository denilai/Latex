\documentclass[a4paper, 12pt]{article}
\usepackage{cmap}					% поиск в PDF
\usepackage[T2A]{fontenc}			% кодировка
\usepackage[utf8]{inputenc}			% кодировка исходного текста
\usepackage[english,russian]{babel}	% локализация и переносы

\usepackage{graphicx}
\graphicspath{ {./}{./1}{./2}{./3} }

\usepackage{caption}
\captionsetup{labelsep=period}
\captionsetup[table]{labelsep=endash,justification=justified,singlelinecheck=false,font=normalsize}

\usepackage{indentfirst}

\usepackage{listings}

\lstdefinestyle{listing_style}{
	extendedchars = \true,
	backgroundcolor=\color{backcolour},
	commentstyle=\color{codegreen},
	keywordstyle=\color{magenta},
	numberstyle=\tiny\color{codegray},
	stringstyle=\color{codepurple},
	basicstyle=\ttfamily\footnotesize,
	breakatwhitespace=false,
	breaklines=true,
	captionpos=b,
	keepspaces=true,
	numbers=left,
	numbersep=5pt,
	showspaces=false,
	showstringspaces=false,
	showtabs=false,
	tabsize=2
}

\lstset{style=listing_style}

\usepackage{xcolor}

\definecolor{codegreen}{rgb}{0,0.6,0}
\definecolor{codegray}{rgb}{0.5,0.5,0.5}
\definecolor{codepurple}{rgb}{0.58,0,0.82}
\definecolor{backcolour}{rgb}{0.95,0.95,0.92}

\date{}\usepackage{graphicx}
\graphicspath{ {./}}

\usepackage{caption}
\captionsetup{labelsep=period}

\usepackage{listings}

\lstdefinestyle{listing_style}{
	extendedchars = \true,
	backgroundcolor=\color{backcolour},   
	commentstyle=\color{codegreen},
	keywordstyle=\color{magenta},
	numberstyle=\tiny\color{codegray},
	stringstyle=\color{codepurple},
	basicstyle=\ttfamily\footnotesize,
	breakatwhitespace=false,         
	breaklines=true,                 
	captionpos=b,                    
	keepspaces=true,                 
	numbers=left,                    
	numbersep=5pt,                  
	showspaces=false,                
	showstringspaces=false,
	showtabs=false,                  
	tabsize=2
}

\lstset{style=listing_style}

\usepackage{xcolor}

\definecolor{codegreen}{rgb}{0,0.6,0}
\definecolor{codegray}{rgb}{0.5,0.5,0.5}
\definecolor{codepurple}{rgb}{0.58,0,0.82}
\definecolor{backcolour}{rgb}{0.95,0.95,0.92}

\usepackage{array}

\title{Практическая работа 3}
\author{Сосунов А.М., группа ИВБО-02-19}
\date{}

\begin{document}
	\maketitle
	
	\begin{enumerate}
		\item Создать чередующийся том через диспетчер дисков, проверить на
		отказ
		
		\item Удалить чередующийся том
		\item Создать зеркальный том через диспетчер дисков, проверить на
		отказ
		\item Удалить отказавшее зеркало, добавить свободный диск в качестве
		зеркала к оставшемуся
		\item Проверить на отказ второго диска после ресинхронизации зеркала
		\item Удалить все тома, преобразовать диски из динамических в
		обычные
		\item Создать простое пространство через «Дисковые пространства»,
		проверить на отказ 1 и 2 дисков
		\item Удалить простое пространство
		\item Создать двустороннее зеркало через «Дисковые пространства»,
		проверить на отказ 1 и 2 дисков
		\item Удалить двустороннее зеркало
		\item Создать пространство с четностью через «Дисковые
		пространства», проверить на отказ 1 и 2 дисков1) Создать чередующийся том через диспетчер дисков, проверить на
		отказ
	\end{enumerate}
\end{document}}