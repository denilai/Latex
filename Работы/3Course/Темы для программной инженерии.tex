% Параграф №2 из труда Эйлера "Новая теория движения Луны" (Эйлер Л. Новая
% теория движения Луны. Перевод с латинского акад. А. Н. Крылова. М.-Л.:
% изд-во АН СССР, 1937. 248 с.).
% Задание выполнено в ходе ознакомления с курсом "Документы и презентации % в LaTex", разработанным Данилом Фёдоровых (danil@fedorovykh.ru),
% записанном НИУ ВШЭ
% для Coursera.org: http://coursera.org/course/latex .

\documentclass[a4paper,12pt]{article} % добавить leqno в [] для нумерации слева

\usepackage{hyperref}

%%% Работа с русским языком
\usepackage{cmap}					% поиск в PDF
\usepackage{mathtext} 				% русские буквы в фомулах
\usepackage[T2A]{fontenc}			% кодировка
\usepackage[utf8]{inputenc}			% кодировка исходного текста
\usepackage[english,russian]{babel}	% локализация и переносы


%%% Работа с картинками
\usepackage{graphicx}  % Для вставки рисунков
\graphicspath{{images/}{images2/}}  % папки с картинками
\setlength\fboxsep{3pt} % Отступ рамки \fbox{} от рисунка
\setlength\fboxrule{1pt} % Толщина линий рамки \fbox{}
\usepackage{wrapfig} % Обтекание рисунков и таблиц текстом

\usepackage {caption}
\usepackage {textcomp}
\usepackage {floatrow}
%\floatsetup[figure]{capposition=bottom, floatwidth=0.9\linewidth, margins=centering}

	
	
% Первая из этих строк запретит все разрывы строк после знаков
% бинарных операций, а вторая — после знаков бинарных отношений, и
% при этом помех верстке абзаца будет меньше, чем при заключении всей
% формулы в фигурные скобки.	
\binoppenalty = 10000
\relpenalty   = 10000


%%% Дополнительная работа с математикой
\usepackage{amsmath,amsfonts,amssymb,amsthm,mathtools} % AMS
\usepackage{icomma} % "Умная" запятая: $0,2$ --- число, $0, 2$ --- перечисление

%% Для использования знака номера 
\usepackage{textcomp}


%% Шрифты
\usepackage{euscript}	 % Шрифт Евклид
\usepackage{mathrsfs} % Красивый матшрифт


%%% Работа с таблицами
\usepackage{array,tabularx,tabulary,booktabs} % Дополнительная работа с таблицами
\usepackage{longtable}  % Длинные таблицы
\usepackage{multirow} % Слияние строк в таблице


%%% Заголовок
\author {К. Денисов}
\title {Идеи для проектов по СиПИ v.0.1}
\date {\today}

\begin{document}
\maketitle

\section*{Web-проекты}
\begin{enumerate}
	\item Социальная сеть для фотографов и моделей.
	\begin{itemize}
		\item поиск взаимовыгодных условий;
		\item выгрузка портфолио;
		\item договоры о сессиях.
	\end{itemize}

	\item Навыки для Алисы.
		\begin{itemize}
			\item анализ и выдача расписания;
			\item сбор сведений о преподавателях и кафедрах;
			\item телефонный справочник.
		\end{itemize}
	\item Анализ открытых данных (например с сайта \href{http://www.data.mos.ru}{Открытых данных Москвы}).
		\begin{itemize}
			\item сбор данных с фитнес браслетов с последующей визуализацией;
			\item подключение Grafana, визуализация и анализ интересных данных;
			\item составление различных топов и подборок по городам и учреждениям;
			\item использование библиотек для работы с данными (обработка python, java  и т.д).
		\end{itemize}
\end{enumerate}
\section*{Игры}
\begin{enumerate}
	\item Игра --- запомни станции московского метро.
	\begin{itemize}
		\item разделение по уровням сложности --- можно угадывать по областям (север, юг, запад, восток), а можно по веткам;
		\item угадывание цветов веток;
		\item интересные факты от метро;
		\item выгрузка данных о метро из открытых данных и использование их в игре.
	\end{itemize}
	\item Игра, основанная на данных.
	\begin{itemize}
		\item пример с социальными или урбанистическими проблемами;
		\item процедурная генерация уровней;
		\item искусственный интеллект.
	\end{itemize}
	\item Цифровизация игр для вечеринок. Сделать базу ведения статистики по играм. Пример игр для цифровизации: серсо, фанты, игры на реакцию (нажать на кнопку), морской бой.
\end{enumerate}

\section*{Аппаратная часть}
\begin{enumerate}
	\item Велокомпьютер.
	\item Работа с Bluetooth Low Energy метками, nfс метками.
	\item Создание устройства ввода (джойстик, использование приводов для управления программами). 
\end{enumerate}


\section*{Программная часть}
\begin{enumerate}
	\item Мощная cli программа для сбора информации из личного кабинета студента, успеваемости, посещаемости.
	\item Программа для анализа затрачиваемых ресурсов (анализ+ визуализация).
	\item Умный дом + веб сервер для работы с ним.
		\begin{itemize}
			\item ESP с wifi (web-server) и взаимодействие с устройствами;
			\item Arduino + модули к нему.
		\end{itemize}
\end{enumerate}

\end{document}
