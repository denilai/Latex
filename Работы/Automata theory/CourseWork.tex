\documentclass[a4paper,14pt]{extarticle}

% Путь до папки с общими шаблонами
\newcommand{\pathToCommonFolder}{../../Common}
% Название работы в титуле
\newcommand{\workname}{Отчет по практическим работам}
% Название дисциплины в титуле
\newcommand{\discipline}{Теория автоматов}
% Название кафедры в титуле
\newcommand{\kafedra}{Кафедра Вычислительной техники}
% Тема работы в титуле
\newcommand{\theme}{Сложение чисел с плавающей точкой}
% Должность преподавателя в титуле
\newcommand{\rang}{ассистент}
% ФИО преподавателя в титуле
\newcommand{\teacherfio}{А.~С.~Боронников}


\usepackage{tabularx}

\usepackage{booktabs}
\newcolumntype{b}{X}
\newcolumntype{s}{>{\hsize=.5\hsize}X}
\newcommand{\heading}[1]{\multicolumn{1}{c}{#1}}

\usepackage{extsizes} % Возможность сделать 14-й шрифт

% Вставка заготовки преамбулы
% Этот шаблон документа разработан в 2014 году
% Данилом Фёдоровых (danil@fedorovykh.ru) 
% для использования в курсе 
% <<Документы и презентации в \LaTeX>>, записанном НИУ ВШЭ
% для Coursera.org: http://coursera.org/course/latex .
% Исходная версия шаблона --- 
% https://www.writelatex.com/coursera/latex/5.3

% В этом документе преамбула

% Для корректного использования русских символов в формулах
% пакеты hyperref и настройки, связанные с ним, стоит загуржать
% перед загрузкой пакета mathtext



% поддержка русских букв
% кодировка шрифта
%\usepackage[T2A]{fontenc} 
\usepackage{pscyr}

% использование ненумеровонного абзаца с добавлением его в содержаниеl

\newcommand{\anonsection}[1]{\section*{#1}\addcontentsline{toc}{section}{#1}}
\newcommand{\sectionunderl}[1]{\section*{\underline{#1}}}


% настройка окружения enumerate
\usepackage{enumitem}
\setlist{noitemsep}
\setlist[enumerate]{labelsep=*, leftmargin=1.5pc}

\usepackage{hyperref}

% сначала ставить \usepackage{extsizes} % Возможность сделать 14-й шрифт
% для корректной установки полей вставлять преамбулу следует в последнюю очередь (но перед дерективой замены \rmdefault)
\usepackage[top=20mm,bottom=25mm,left=35mm,right=15mm]{geometry} % Простой способ задавать поля

\hypersetup{				% Гиперссылки
	unicode=true,           % русские буквы в раздела PDF
	pdftitle={Заголовок},   % Заголовок
	pdfauthor={Автор},      % Автор
	pdfsubject={Тема},      % Тема
	pdfcreator={Создатель}, % Создатель
	pdfproducer={Производитель}, % Производитель
	pdfkeywords={keyword1} {key2} {key3}, % Ключевые слова
	colorlinks=true,       	% false: ссылки в рамках; true: цветные ссылки
	linkcolor=red,          % внутренние ссылки
	citecolor=black,        % на библиографию
	filecolor=magenta,      % на файлы
	urlcolor=blue           % на URL
}

%%% Работа с русским языком
\usepackage{cmap}					% поиск в PDF
\usepackage{mathtext} 				% русские буквы в формулах
\usepackage[T2A]{fontenc}			% кодировка
\usepackage[utf8]{inputenc}			% кодировка исходного текста
\usepackage[english,russian]{babel}	% локализация и переносы
\usepackage{indentfirst}
\frenchspacing

%для изменения названия списка иллюстраций
\usepackage{tocloft}


\renewcommand{\epsilon}{\ensuremath{\varepsilon}}
\renewcommand{\phi}{\ensuremath{\varphi}}
\renewcommand{\kappa}{\ensuremath{\varkappa}}
\renewcommand{\le}{\ensuremath{\leqslant}}
\renewcommand{\leq}{\ensuremath{\leqslant}}
\renewcommand{\ge}{\ensuremath{\geqslant}}
\renewcommand{\geq}{\ensuremath{\geqslant}}
\renewcommand{\emptyset}{\varnothing}

% Изменения параметров списка иллюстраций
\renewcommand{\cftfigfont}{Рисунок } % добавляем везде "Рисунок" перед номером
\addto\captionsrussian{\renewcommand\listfigurename{Список иллюстративного материала}}

\newcommand{\tm}{\texttrademark\ }
\newcommand{\reg}{\textregistered\ }




%требования к спискам, возможно пригодится
%---------------------------------------------------------
%Требования на списки в стандарте следующие:
%нумерованные списки на первом уровне помечаются как «а)», «б)», «в)»… На втором — как «1)», «2)», «3)». Да-да, я тоже не вижу тут ни капли логики.
%ненумерованные списки помечаются дефисами.

%\usepackage{enumitem}
%\makeatletter
%\AddEnumerateCounter{\asbuk}{\@asbuk}{м)}
%\makeatother
%\setlist{nolistsep}
%\renewcommand{\labelitemi}{-}
%\renewcommand{\labelenumi}{\asbuk{enumi})}
%\renewcommand{\labelenumii}{\arabic{enumii})}
%------------------------------------------------------


%%% Дополнительная работа с математикой
\usepackage{amsmath,amsfonts,amssymb,amsthm,mathtools} % AMS
\usepackage{icomma} % "Умная" запятая: $0,2$ --- число, $0, 2$ --- перечисление

%% Номера формул
%\mathtoolsset{showonlyrefs=true} % Показывать номера только у тех формул, на которые есть \eqref{} в тексте.
%\usepackage{leqno} % Нумереация формул слева

%% Свои команды
\DeclareMathOperator{\sgn}{\mathop{sgn}}

%% Перенос знаков в формулах (по Львовскому)
\newcommand*{\hm}[1]{#1\nobreak\discretionary{}
{\hbox{$\mathsurround=0pt #1$}}{}}


% отступ для первого абзаца главы или параграфа
%\usepackage{indentfirst}

%%% Работа с картинками
\usepackage{graphicx}  % Для вставки рисунков
\graphicspath{{images/}{screnshots/}}  % папки с картинками
\DeclareGraphicsExtensions{.pdf,.png,.jpg}
\setlength\fboxsep{3pt} % Отступ рамки \fbox{} от рисунка
\setlength\fboxrule{1pt} % Толщина линий рамки \fbox{}
\usepackage{wrapfig} % Обтекание рисунков текстом

%%% Работа с таблицами
\usepackage{array,tabularx,tabulary,booktabs} % Дополнительная работа с таблицами
\usepackage{longtable}  % Длинные таблицы
\usepackage{multirow} % Слияние строк в таблице

%%% Теоремы
\theoremstyle{plain} % Это стиль по умолчанию, его можно не переопределять.
\newtheorem{theorem}{Теорема}[section]
\newtheorem{proposition}[theorem]{Утверждение}

\theoremstyle{plain} % Это стиль по умолчанию, его можно не переопределять.
\newtheorem{work}{Практическая работа}[part]


 
 
\theoremstyle{definition} % "Определение"
\newtheorem{corollary}{Следствие}[theorem]
\newtheorem{problem}{Задача}[section]
 
\theoremstyle{remark} % "Примечание"
\newtheorem*{nonum}{Решение}



%%% Программирование
\usepackage{etoolbox} % логические операторы

%%% Страница

%	\usepackage{fancyhdr} % Колонтитулы
% 	\pagestyle{fancy}
%   \renewcommand{\headrulewidth}{0pt}  % Толщина линейки, отчеркивающей верхний колонтитул
% 	\lfoot{Нижний левый}
% 	\rfoot{Нижний правый}
% 	\rhead{Верхний правый}
% 	\chead{Верхний в центре}
% 	\lhead{Верхний левый}
%	\cfoot{Нижний в центре} % По умолчанию здесь номер страницы

\usepackage{setspace} % Интерлиньяж
\onehalfspacing % Интерлиньяж 1.5
%\doublespacing % Интерлиньяж 2
%\singlespacing % Интерлиньяж 1

\usepackage{lastpage} % Узнать, сколько всего страниц в документе.

\usepackage{soul} % Модификаторы начертания


\usepackage[usenames,dvipsnames,svgnames,table,rgb]{xcolor}


\usepackage{csquotes} % Еще инструменты для ссылок

%\usepackage[style=authoryear,maxcitenames=2,backend=biber,sorting=nty]{biblatex}

\usepackage{multicol} % Несколько колонок

\usepackage{tikz} % Работа с графикой
\usepackage{pgfplots}
\usepackage{pgfplotstable}

% модуль для вставки рыбы
\usepackage{blindtext}

\usepackage{listings}
\usepackage{color}


% для поворота отдельной страницы. Использовать окружение \landscape
\usepackage{pdflscape} 
\usepackage{rotating} 


\definecolor{mygreen}{rgb}{0,0.6,0}
\definecolor{mygray}{rgb}{0.5,0.5,0.5}
\definecolor{mymauve}{rgb}{0.58,0,0.82}


% пример импорта файла
%\lstinputlisting{/home/denilai/repomy/conf/distributions}

\lstset{
	language=Python,
	basicstyle=\footnotesize,        % the size of the fonts that are used for the code
	numbers=left,                    % where to put the line-numbers; possible values are (none, left, right)
	numbersep=5pt,                   % how far the line-numbers are from the code
	numberstyle=\tiny\color{mygray}, % the style that is used for the line-numbers
	stepnumber=2,                    % the step between two line-numbers. If it's 1, each line will be numbered
	% Tab - 2 пробела
	tabsize=2,    
	% Автоматический перенос строк
	breaklines=true,
	frame=single,
	breakatwhitespace=true,
	title=\lstname 
}



% установка размера шрифта для всего документа
%\fontsize{20pt}{18pt}\selectfont


\author{Кирилл Денисов}
\title{Практическая работа №5}
\date{\today}

% установка полуторного интервала
% \usepackage{setspace}  
% \onehalfspacing

% использовать Times New Roman
\renewcommand{\rmdefault}{ftm}


\begin{document}
	\thispagestyle{empty}
	
	% Вставка первого титульного листа
	%%\newcommand{\withouttheme}{} добавить эту переменную для определения, нужна ли тема
%     {} - нужна
%    {1} - не нужна

%\newcommand{\withoutsubmissiondate}{} добавить эту переменную для определения, нужен ли срок предоставления отчета
%     {} - нужен
%    {1} - не нужен

\renewcommand{\studentfio}{К.~Ю.~Денисов\\
			%	& & \hfill И.~А.~Кремнев \\
				& & \hfill А.~М.~Сосунов\\
				& & \hfill Д.~Н.~Федосеев}

\begin{center}
	\begin{figure}[h!]
		\begin{center}
			%\vspace{10ex}
			\includegraphics[width=0.17\linewidth]{\pathToCommonFolder/gerb}
			%\caption{}\label{pic:first}
			%	\vspace{5ex}
		\end{center}	
	\end{figure}
	\small	МИНОБРНАУКИ РОССИИ \\
	Федеральное государственное бюджетное образовательное учреждение\\
	высшего образования\\
	\normalsize					
	\textbf{«МИРЭА – Российский технологический университет»\\
		РТУ МИРЭА}\\
	\noindent\rule{1\linewidth}{1pt}\\
	Институт информационных технологий\\ %\vspace{2ex}
	\kafedra\\
	\vspace{3ex}
	\large \textbf{\workname}  \\
	%\vspace{1ex}
	по дисциплине\\ «\discipline» \\
	\vspace{3ex}
	\if \withouttheme
	\textbf{Тема работы:}\\ <<\theme>>
	\fi
	\vspace{6ex}
	\small
	\begin{table}[h!]
		\begin{tabular}{lp{0.6\linewidth}l}
			\textbf{Выполнили:} & студенты группы ИВБО-02-19 & \\ 
			& & \hfill \studentfio \\%Д.~Н.~Федосеев\\%А.~М.~Сосунов\\%К.~Ю.~Денисов\\%И.~А.~Кремнев
			\textbf{Принял:} & \rang & \\
			& & \hfill \teacherfio\\
		\end{tabular}
	\end{table}
	\if \withoutsubmissiondate
	\begin{flushleft}
		Отчет представлен <<\rule{3ex}{1pt}>>\rule{10ex}{1pt} 202\rule{1ex}{1pt} г.
	\end{flushleft}
	\fi
	\normalsize
	
	\vfill
	Москва 2021
	
\end{center}
	
	\newpage
	\tableofcontents
	%\listoffigures
	\newpage
	
\section*{Расчетно-пояснительная записка}
\section{Постановка задачи}
Разработать вычислительное устройство, состоящее из  двух взаимосвязанных частей --- операционного и управляющего автоматов, и выполняющее следующие операции:
\begin{enumerate}
	\item Деление двух целых чисел в дополнительном коде;
	\item Сложение двух чисел, представленных в экспоненциальном формате.
\end{enumerate}
Операнды представлены в виде 32 двоичных разрядов. Управляющий автомат реализовать по схеме с регулярной адресацией в последовательном варианте.
\section{Интерфейс устройства}
\section{Описание алгоритмов}
\subsection{Деление двух целых чисел в дополнительном коде}
Для деления двух целых чисел представленных в двоичном дополнительном коде реализуем алгоритм деления без восстановления остатка. Данный алгоритм является оптимальным по суммарному времени, так как обработка очередного разряда результа осуществляется за один такт.

Введем обозначения операндов, используемых в данной операции (таблица \ref{tab:vars}):
\begin{table}[h!]
	\centering
	\begin{tabular}{|c|c|}
		\hline
		\multicolumn{1}{|c|}{\textbf{Обозначение}} & \multicolumn{1}{c|}{\textbf{Назначение}} \\ \hline
		A & Делимое в доп. коде \\ \hline
		B & Делитель в доп. коде \\ \hline
	\end{tabular}
	\label{tab:vars}
	\caption{Операнды}
\end{table} 
\newpage
Опишем алгоритм деления следующим образом:
\begin{enumerate}
	\item \label{enum:first} Анализируем знак делимого --- если делимое отрицательное, то все биты регистра частичного остатка устанавливаем в <<1>>, если делимое положительное, то все биты регистра частичного остатка устанавливаем в <<0>>;
	\item Сдвигаем остаток влево; на место младшего разряда помещаем старший разряд делимого;
	\item Анализируем знаки остатка и делителя --- в случае, если их знаки одинаковые, то выполняем вычитание делителя из остатка (прибавляем противоположное число), полученного на данном этапе. Иначе же прибавляем значение делителя к значению остатка%
	\item \label{enum:last} Анализируем значение остатка после выполнения арифметических действий --- заносим в частное инвертированный знак остатка, вычисленного на данном этапе, вместе с этим сдвигая его влево. 
	\item Повторяем пункты \ref{enum:first}-\ref{enum:last} до тех пор, пока не будут сдвинуты все разряды делимого. 
\end{enumerate}

Стоит отметить, что для формирования правильного выходного результата после выполнения вышеперечисленных пунктов необходимо выполнить коррекцию значений частного и остатка в зависимости от знаков операндов. Для каждой комбинации знаков делимого и делителя реализована отдельная операция коррекции. См таблицу \ref{tab:correction4}.

\begin{table}[h!]
	\centering
	\begin{tabular}{|m{0.2\linewidth}|m{0.67\linewidth}|}
		\hline
		\textbf{Комбинация} &\multicolumn{1}{c|}{\textbf{Коррекция}}\\
		\hline
		$A\ge0, B>0 $ & Коррекция не требуется \\ 
		\hline
		$A\ge0, B<0$ & Перевести частное в доп. код\\
		\hline
		$A\le0, B>0$ & Результат верен, если остаток $=0$. Иначе прибавить к отрицательному частному единицу, перевести остаток в доп. код\\
		\hline
		$A\le 0, B\le 0$ & Изменить знак делимого, перевести остаток в доп. код\\
		\hline
	\end{tabular}
	\caption{Коррекция результата}
	\label{tab:correction4}
\end{table}
\newpage

Приведем пример вычисления частного от деления чисел $(-13_{10}=10011_2)\div 3_{10}=00011_2$. См. таблицу \ref{tab:examplediv}.
\begin{table}[h!]
	\small

	\begin{tabular}{|l|l|l|m{0.6\linewidth}|}
		\hline
		\multirow{2}*{Частное} & Остаток & Делимое & 	\multirow{2}*{Операция} \\ \cline{2-3}
		& \multicolumn{1}{l|}{11111} & \multicolumn{1}{l|}{10011} &  \\ \hline \hline
		& \multicolumn{1}{l|}{11111} & 0011х & Сдвиг остатка \\ \hline
		& 00011 &  & Сложение с делителем \\ \hline
		1 & 00010 &  & Результат сложения — положительный остаток \\ \hline
		& 00100 & 011хх & Сдвиг остатка \\ \hline
		& 11101 &  & Вычитание делителя \\ \hline
		1 & 00001 &  & Результат вычитания — положительный остаток \\ \hline
		& 00010 & 11ххх & Сдвиг остатка \\ \hline
		& 11101 &  & Вычитание делителя \\ \hline
		0 & 11111 &  & Результат вычитания — отрицательный остаток \\ \hline
		& 11111 & 1хххх & Сдвиг остатка \\ \hline
		& 00011 &  & Сложение с делителем \\ \hline
		1 & 00010 &  & Результат вычитания — положительный остаток \\ \hline
		& 00101 & ххххх & Сдвиг остатка \\ \hline
		& 11101 &  & Вычитание делителя \\ \hline
		1 & 00010 &  & Результат вычитания — положительный остаток \\ \hline
		& 11111 &  & Восстановленный отрицательный остаток \\ \hline
	\end{tabular}
	\caption{Пример деления целых чисел в доп. коде}
	\label{tab:examplediv}
\end{table}

В результате вычислений получим частное $(-5_{10}) = 11011_2$ и остаток $2_{10}=00010_2$.

Опишем алгоритм работы автомата с помощью блок схемы. Используем сумматор для нахождения текущего значение частичного остатка (ЧО), счетчик для подсчета обработанных разрядов и регистры для хранения и использования разрядов делителя и делимого. Обозначим микрокоманды от $m_0$ до $m_4$. См. рис.~\ref{img:algorithm4}.
\begin{figure}[h!]
	\centering
	\includegraphics[width=0.5\linewidth]{algorithm3}
	\caption {Алгоритм деления двух чисел в дополнительном коде}
	\label{img:algorithm4}
\end{figure}
\subsubsection{Сложение двух чисел в экспоненциальном формате}

\end{document}


